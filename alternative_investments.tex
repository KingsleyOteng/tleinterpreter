\documentclass[12pt]{article}
\newlength{\rulelength}
\setlength{\rulelength}{15cm}
\usepackage[left=2cm, right=2cm, top=2cm]{geometry}
\usepackage[utf8]{inputenc}
\usepackage{setspace}
\usepackage{amsmath,amsthm,amscd}
\usepackage[colorlinks]{hyperref}
\usepackage{setspace}
\usepackage{xcolor}
\usepackage{xparse}
\usepackage{soul}
\usepackage{graphicx}
\usepackage{framed}

\begin{document}
\newpage

\textbf{Determining payoff from an investment in real estate using one's own money?}

PV =  - the amount of ones own cash in investment
N = holding period
PMT = NOI - Debt Servicing
FV = Selling price - Debt Size
Solve for I

\section{Alternative Investments}

\textbf{How do we calculate the gross less equivalent?}

gross less equivalent = net lease + other expenses

\textbf{What is generally asociated with VC's?}

1) Weak balance sheet
2) New management

\textbf{What is a disadvantage between REITs and private real estate?}

1) Principal agent problem. 
2) REITs will not pass on the tax losses

\textbf{What is a disadvantage between REITs and private equity real estate?}

No clawbacks in REITs

\textbf{How does private equity create value?}

1) Direct control over management
2) Reengineer a company
3) Better alignment of manager interests

\textbf{Where does the risk-return profile sit for private equity?}

Between corporate debt and equities

\textbf{How are returns on private equity investments most likely to be reported?}



\textbf{Percentage of sale leases?}

More common for large retail setups or just retail

\textbf{Net lease?}

Requires that an individual pay part of the taxes, fees, maintenance costs in addition to their actual rent. Typically for industrial properties (not mini warehouse).  Net leases are more common for single tenant uses, smaller retail developments, and warehouses; 



\textbf{What is a gross lease}

The owner only pays rent - none of the taxes. 

\textbf{Why do owners prefer a 'net lease'?}

Less volatility in the income stream

\textbf{What is a property specific concern - when investing?}

Demographic shifts

\textbf{What is the primary driver for multi-family housing?}

Demographic shifts. Not GDP or economic activity.

\textbf{What is due diligence most likely to deal with - from the stand point of buying a chemcial factory? Government regulations?}

No - uncovering the use of a contaminant that may prevent its desired use. 

\textbf{What will due dilligence typucally cover?}

Everything outside of the contracts, market analysis and government regulation.  For example - management skills, tenancy concentration, lease terms

\textbf{What is the difference between a DOWNREIT and an UPREIT?}

UPREIT: REIT is the GP
DOWNREIT: REIT as just an LP

\textbf{How should you deal with the all risk yield?}

When the all risk yield becomes effective - ignore the other yields. 

\textbf{Which measure is mostly likely used for PE funds? TWR, MWR and IRR?}

IRR

\textbf{What is the ROEfor AFFO given: P/NAV = 0.97, P/AFFO = 12.3, AFFO Payout = 0.65 and AFFO growth?}

$$
g_{AFFO growth} = ROE / (1 - PR_{AFFO Payout})\\
ROE = g_{AFFO growth} \times(1 - PR_{AFFO Payout})\\
$$

\textbf{How do we calculate NAV wwrp to market debt?}

NAV = (NOI / (caprate)) - Debt

\textbf{How do we define the caprate?}

Caprate = Discount Rate - Growth Rate

\textbf{How do we define the caprate from comparables?}

Caprate = NOI / ()Comparables Sale Price)

\textbf{What is the All Risk Yield (ARY)?}

ARY = Rent / Value

\textbf{Waht is the difference in the value calculation in the 1) direct capitalization method and 2) DCF Method?}

1) growth is implicit in the direct capitalization method

Value_DC = NOI / Caprate

2) growth is explicit in the DCF method

Value_DCF = NOI / (r - g)

\textbf{What is the biggest difference beween owning a bond and owning a commodity?}

Bonds have no economic value

\textbf{What is a ratchet provision?}

It is a trigger in contracts that provide managers with a greater proportion of profits the more successful they are

\textbf{What is a cash crop?}

A crop that is not considered a staple of our diet. Such as coffee and cocoa

\textbf{What is a basis swap?}

Receiving the credit spread from CDS versus paying the spread from the corporate bond market

\begin{framed}
	
	\textbf{What is a term sheet used for?}
	
	It is used to align the interest of management and the private equity investors
	
	\textbf{Term sheet item: earnouts?}
	
	Earn outs link the acquisition price paid to the owner to a multiple of the firm's performance over 2–3 years. The owner therefore has an incentive to meet performance milestones in order to maximize acquisition price.
	
	\textbf{Term sheet item: preferred dividends?}
	
	Preferred dividends apply primarily to non-VC private equity transactions and non-compete clauses may apply to either VC or other private equity transactions.
	
	\textbf{Discuss PE and VC with respect to burn rate, milestone achievement and returns
		
		VC - high burn rate
		VC - Milestone based achievement
		PE - returns
	
	\textbf{What type of 'equity' are REITs?}
	
	public equity
\end{framed}

\newpage

\textbf{How do we calculate 'NOI'?}

NOI = rental income + other income - vacancy losses - management fee - other expenses

\textbf{How do we calculate the terminal value of an investment}

Solve for:

$$
Terminal Value = \dfrac{Terminal NOI}{Terminal Caprate}
$$

PMT = Regular NOI, FV = Terminal Value, N= Holding Period, I/Y = DF


\textbf{Where does NOI sit relative to tax rate?******}

It's a pre-tax caculation

\textbf{Where does NOI sit relative to interest expense?******}

It's a pre interest expense calculation

\textbf{Does NOI consider the tax rate and interest expense?}
NO. NOI is similar to an EBIT

\textbf{Net Operating Income (NOI) calculation}

NOI = Rental Income + Other Operating Income $\times (\text{Operating expenses} + \text{Property management fee})$

\textbf{Value of a renovated property}

\begin{equation}
\begin{split}
Value  & = V_{\text{renovated}} - V_{\text{loss prior to renovation}} \\
Value  & = \dfrac{\text{NOI}(1+g)}{r - g} - \dfrac{(\text{NOI}  -  \text{NOI}_{old})}{1 + r}
\end{split}
\end{equation}

\textbf{What is the equivalent of discount rate for real estate?}

cap rate

\textbf{How do you account for loss of usable life for an asset?}

Value - Incurable losses (1) - Curable losses (2)
$$
\text{Redundant Value} = (\text{Value} - \text{sum of incurable structural defects}) \Biggr( \dfrac{\text{Years Lost + Years Expired}}{\text{Regular Life-span}} \Biggl)
$$

\textbf{How do you account for loss of usable life for an asset?}

\begin{equation}
\text{Redundant Value} = (\text{Value} - \text{sum of incurable structural defects}) \Biggr( \dfrac{\text{Years Lost + Years Expired}}{\text{Regular Life-span}} \Biggl) + \text{Land Value}
\end{equation}

\textbf{Value of a property?}

1) Sum incurable defects
2) Sum all curable
3) Sum incurable NOI by Cap rates
4) Add land 
\textbf{Highest and best use of land is:}

The project which generates the highest NPV.

\textbf{What's the difference between REOC's and REIT's?}
1) REOC's are more flexible than REIT's - the requirements are not
2) REIT's have to payout 90 percent to 95 percent of their income (to maintain tax-free status)
3) REITS have to be about 90 percent invest in real estate

\textbf{What is the best valuation method for REITs?  **************}

NAV. Not market value or book value

\textbf{Which of the following techniques would be best applied to value an early stage company? ************}

Replacement cost

\textbf{Which of the following techniques would be best applied to an established company? ************}

Income approach or relative valuation approach.

\textbf{What factors drive REIT changes?}

High-level macroeconomic factors

\textbf{Which of the following economic drivers is the most important when considering the factors affecting storage REITs? 1) National GDP growth, 2) New space versus supply demand, 3) Population growth}

The high-level factor of National GDP growth

\textbf{How much of a REIT's property income can be invested outside property?}
about 5 to 90 percent

\textbf{How do REITs grow if they distribute so much income?}
From secondary offerings

\textbf{What is 'AFFO'?}


\textbf{out of AFFO, FFO and NOI  is most like FCFE? ********}

AFFO = FCFE
FFO = NOI - G/A losses or expenses - interest expense

\begin{framed}
\textbf{Maximum debt service}

Maximum Debt Service = NOI / (Debt Service Credit Ratio)

\textbf{Loan amount from DSCR}

Loan amount = NOI / (Debt Service Credit Ratio $\times$ Mortgage Rate)

\textbf{Loan amount from maximum-loan-to-value}

Loan amount =  (NOI $\times$ loan multiple $\times$ l maximum-loan-to-value) 

\textbf{Maximum loan amount}

Minimum of amount from DSCR or other means.

\end{Framed}

\textbf{What are the pecularities of NOI}

1) It does not include goodwill
2) it does not include DTA and DTL
3) Noncash rents are deducted / not included

\textbf{In terms of the risk of asset classes of bonds, real estate and stocks - which are the most risky to least risky?}

stocks, real estate and then bonds

\textbf{Management risk with respect to private equity commercial real estate investments }

In refers to managing the risk in both the real estate itself portfolio and the asset class

\textbf{What is the difference between the capitalization approach and the DCF approach from the perspective of NOI?}

1) The DCF approach allows for variability in NOI
2) The Caprate approach assumes a constant NOI

\textbf{What is the relationship between TVPI, DPI and RVPI}

TVPI = DPI + RVPI

\textbf{How do you interpret TVPI, DPI and RVPI? **********}

TVPI = the total expected return on investment
DPI = the total return on ivnestment thus far
RVPI = the unrealized return on investment

\textbf{How do you interpret a DPI of 0.37?}

That the fund has returned 37 cents for every dollar invested so far. 

\textbf{What is another name for RVPI? *******}

NAV

\textbf{What is the PIC?}

Paid-in-to-committed-capital ratio. You want this to be low

\textbf{How do you determine when an investment is still undergoing the J-curve effect?*********}

It will have a negative Net IRR

\begin{framed}
	\textbf{DPI, RVPI and TVPI}\\
	\newline
	
	\begin{framed}
		\textbf{DPI}\\
		\begin{equation*}
		DPI = \dfrac{\sum^T_{t=0} D_t}{\sum^T_{t=0} C_t}
		\end{equation*}\\
		where $C_t$ is the cumulative invested capital, $D_t$ are the dividends at time $t$ and $NAV_T$ is the net asset value at time $T$.
		
	\end{framed}
	
	
	
	\begin{framed}
		\textbf{RVPI}\\
		\begin{equation*}
		RVPI = \dfrac{NAV_T}{\sum^T_{t=0} C_t}
		\end{equation*}\\
		where $C_t$ is the cumulative invested capital and $NAV_T$ is the net asset value at time $T$.
	\end{framed}
	
	
	
	
	\begin{framed}
		\textbf{TVPI}\\
		\begin{equation*}
		TVPI = \dfrac{\sum^T_{t=0} D_t + NAV_T}{\sum^T_{t=0} C_t}
		\end{equation*}\\
		where $C_t$ is the cumulative invested capital, $D_t$ are the dividends at time $t$, $C_t$ are the capital investments at time $t$ and $NAV_T$ is the net asset value at time $T$.
	\end{framed}
\end{framed}

\begin{figure}
		\includegraphics[width=0.5\textwidth]{contango-backwardation.eps}{ \\
			Contango and Normal-Backwardation }
\end{figure}

\textbf{Classify the following types of real estate (shopping area, restaurant,  hotel,  theme park, restaurants, universities, colleges, parking garages, recreational uses)****}
other = Restaurants, universities and colleges, parking garages, recreational uses, hospitals
hospitality
retail = malls

	
\begin{framed}
	
	\textbf{insurance theory}: 
	
	$F < S_0$
	
	states that investors, farmers, are willing to invest in a future below the spot price, to hedge their crops. The expect that the two prices will converge in the future and they will make a profit that way.
	
	\textbf{hedging pressure hypothesis}: 
	$F > S_0$
	
	Commodity consumers attempt to lock-in a better future price now and commodity production look to hedge their future harvest.
	
	1) Contango: But there are more commodity consumers than commodity producers - hence the market will initially be in contango. The speculators drive it back to convergence. 
	
	2) Normal backwardation:  But when there are more commodity producers than commodity consumers -  the market will initially be in normal backwardation. The speculators drive it back to convergence. 
	
	\textbf{Convenience yield} 
	
	the added convenience of having the commodity available when required. It is what is earned in holding the commodity rather than being long in futures. 
		
	\textbf{theory of storage}:
	
	1) Storage costs can be avoided: producers will enter the market and will purchase futures at lower than their spot simply to lock-in profits. $S_0 > F$
	2) Storage costs can not be avoided: producers will transfer the costs to consumers, consumers are willing to pay a higher price for guaranteed delivery. $F > S_0$
	3) Convenience yield: when consumers prefer to store the commodity on hand this puts downward pressure on the price
	
	Future Price = Spot Price + Storage Costs - Convenience Yield
	
	\textbf{Valuing a Forward}
	
	$$
	F = S_0 e^{r + c - y}
	$$
	
	where $r$ is the risk free rate, $c$ is the convenience yield and $s$ is the storage costs.
	
	\textbf{What are the three components of return for a fully collaterized future?}
	
	(1) Collateral return - returns due to the changes in interest rates over time
	(2) Risk free return 
	(3) Price return
	(4) Roll return
	
	\textbf{What does a speculator earn a profit for rolling a futures contract?}
	
	1) When the commodity curve is in normal-backwardation
	2) the investor only needs the same number of contracts to be rolled
	3) they pocket the difference in price
	
	\begin{framed}
		\section{Types of swap contracts}
		
		\section{Standard Swap Contract}: 
		1) exchange fixed for floating payments
		
			\section{Excess Return Swap Contract}: 
		1) both parties agree to exchange premiums above a certain reference rate
		
		\section{Total Return Swap Contract}
		1) One party pays the return on a commodity index
		2) The other party pays a money market return plus a spread
		
		\section{Basis Swap}
		Is the difference say in commodities, between the spot rate of a commodity and the futures rate of a commodity.
		
		\section{Variance Swap}
		
	\end{framed}

\section{What type of products do commodity indexes prefer to hold?}

Higher value commodities.

\end{framed}

\newpage

\textbf{What is the 'POST value'}

POST-value = PV of Terminal Value

\textbf{Number of shares VC requires to complete a transaction}

\begin{equation}
(\text{Percentage Owned By VC}) = \dfrac{ (\text{Shares_new})}{(\text{Shares_old + Shares_new})}
\end{equation}

\begin{framed}

\textbf{Discount factor for success-or-failure used by VC's?}

\begin{equation}
\begin{split}
r_{s+f} & = \dfrac{P_s + P_f}{1 - P_f} 
\end{split}
\end{equation}

where $r_{s+f}$ is the 'success-failure' discount rate, $P_s$ is the probability of success and $P_f$ is the probability of failure.

\textbf{Value of tabled project in VC community?}

\begin{equation}
VC = \dfrac{\text{targetted Exit Value}}{(1 + r_{s+f})^{time-to-exit}}
\end{equation}


\textbf{Adjusted discount rate for failure only? ************}

$$
r_{adjusted} = \dfrac{1 + DF_0}{1 - failure} - 1
$$

where $DF_0$ is the original discount rate/SYSTEMIC RISK and $f$ is the probability of failure

\textbf{How do we value an early-stage company: 1) the income approach, 2) relative valuation and 3) replacement cost?}

Income approach - no income
Relative approach - no market estimates
Replacement costs - best option

\textbf{What is the best way to value a REIT?}

Net Asset Value (NAV)

\textbf{Number of shares which need to be issued?}

\begin{framed}

$$
	\dfrac{\text{Present Value of Investment}}{\text{PV Value of EXIT}} = \dfrac{\text{No. of Shares to issue}}{\text{No. of Shares to issue} + \text{No. of shares issued} }
$$

\end{framed}

\end{framed}

\textbf{How do we solve the cap-rate?}

$$
Cap-rate  = \dfrac{NOI of comparable}{Total value of comparable}
$$


\textbf{How do we value a real estate property purchased with a loan?}

Value of property = Debt Value + Equity Value

$$
Debt Value = Size of Loan
$$

$$
Equity Value = \dfrac{\text{NOI} - \text{Loan repayments}}}{\text{EYR}}
$$

Therefore Total Value = Equity Value + Debt Value

\textbf{How do we define the equity yield rate (EYR)?}

\begin{equation}
\text{EYR} = \dfrac{\text{Pre-text cashflow}}{\text{Equity}}
\end{equation}

\textbf{Valuing equity from NOI when purchasing with debt?}

Value of Equity Component / Pre-tax cashflow = NOI - (loan rate $\times$ Overall cost)

Equity = Value of Equity Component / EYR

\textbf{Compare, Funds from operations (FFO) and Adjusted funds from operations (AFFO)?}

FFO  	 = Net Earnings + DTA + D \pm (Sales + Restructuring)
AFFO	= Net Earnings + DTA + D \pm (Sales + Restructuring) - (Noncash payments) - (Maintenance and leasing costs)

where DTA is a deferred tax asset and D is depreciation.


\textbf{FFO from NOI}

FFO = NOI - GAE - IE

where NOI is the net operating income, GAE is the general and administrative expenses and IE is the interest expense.


\textbf{AFFO from NOI}

AFFO = FFO - NCR - LC - MC

where NCR is the non-cash rent, LC is the leasing commission and MC is the maintenance capex.


\textbf{True-up}: is the annual calculation of clawbacks as opposed to the the end of year. 

\textbf{Natural position [in spot markets]} is where a producer enters the market to hedge their produce; these firms typically are already active in the market from a producer side.

\textbf{Natural position} = hedger

\textbf{Carried interest calculation}

Carried Interest = (Proceeds - Investment) $\times 20\%$

\textbf{Total Carried Interest}

1) Determine company's that have beaten the hurdle rate
2) Determine the individual carried interest

\textbf{Is carried interest calculatted on a post-distribution or pre-distribution measure?}

A pre-distribution measure


\textbf{How should the 'basis' reflect to the spot price?**********}

It should always be subtracted

\textbf{What is the relation between spot price, future price and the basis ******?}

Futures price = Spot price - basis 

$F = S e^{(r + c - \lambda)T}$

where $r$ is the risk free rate, $c$ is the storage costs and $\lambda$ is the dividend rate.

\textbf{What is the relation between spot price, future price and the basis?}

$F = S . e^{r + c - y}$

\textbf{Describe the (1) Cost Approach, (2) Income Approach and (3) Comparables Approach?}

\textbf{When are you to use the (1) Cost Approach, (2) Income Approach and (3) Comparable Approach?  *****************}

If properties are actually selling in the market - use the comparables approach

\textbf{How should we view the frequency of 1) appraisals and 2) transactions? *****}

1) Appraisals are less frequent 
2) Transactions are frequent

\textbf{in general when should the appraisals method be used? What is the problem with this approach? ******}

1) when transactions are minimal
2) hard to capture capital gains 

\textbf{What type of statistical technique do transaction based indices use? Disadvantage *****?}

Transactions-based indexes run the risk of introducing price fluctuations or random distortions, or “noise,” due to their regression-based derivation.

\textbf{From a rental perspective - what is the DCF method?}

Appraisal

\textbf{Which methods are more and less choppy - appraisal/DCF, repeat sales and hedonic pricing? *****}

1) DCF is choppy 
2) repeat sales - no, as regular transaction flow
3) hedonic pricing - no, as regular transaction flow

\textbf{How should we perceive NAV? *************}

Equtity value from rent at (t+1) plus cash holdings

\textbf{From this calculate the NAVPS: 
	\newline
	Last 12 months' real estate NOI	\$450,000 \\
	Noncash rents	\$50,000  \\
	Next 12 months' growth in NOI	\$10,000  \\
	Cap rate	5\%  \\
	Cash and cash equivalents	\$100,000  \\
	Land held for development	\$150,000  \\
	Accounts receivable	\$10,000  \\
	Prepaid assets	\$5,000  \\
	Total debt	\$3,000,000  \\
	Other liabilities	\$105,000  \\
	Shares outstanding	45,230  \\
}


Value = ( NOI + Growth - NoncashRent ) / growth_rate

NAV = Value + Cash + Land + AR + PA - Debt - Liabilities

NAVPS = NAV / Shares

\textbf{DSCR}

$$
\text{DSCR} = \dfrac{\text{NOI}}{\text{Maximum Debt}}
$$

\textbf{Gordon growth model to a property appreciating}

\begin{equation}
GGM = Value - Loss \\
GGM = \dfrac{V_1}{\text{rr - g}} -  \dfrac{V_1 - V_0}{\text{1 + rr}}
\end{equation}

\textbf{What is 'core real estate'?}

Are considered as having lower risk than other real estate. E.g. warehouses, buildings, restaurants, office, industrial space and multi-family

\textbf{How do you use your calculator to solve for IRR from a property investment using NOI?}

N = years, PV = - initial investment, FV = payoff, PMT = (NOI - debt servicing), i = IRR

\textbf{What makes for a good inflation hedge in property? ***********}

1) indexing mechanically or through regular turnover
2) Otherwise short-term leases work

\textbf{What makes for a poor inflation hedge in property? *******}

1) any interim vacancy
2) longterm leases


\textbf{What types of properties are buyout investors usually looking for?}

1) Established properties
2) Bluechip

\textbf{How should we view NAVPS?}

Profit / (Number of shares)

\textbf{What is the VC's share of ownership if the equity owners hold X?}

VC's ownership = (Y / (Y + X))

\textbf{Equation of collateral return?}

\textbf{In calculating the NAV for a REIT - what is left out?}

1) DTA's
2) DTL's
3) Goodwill
4) Deferred financing expenses

\textbf{How do we calculate the implied value of reusing a site with an existing project?  **********}

Value = Completed Value - Teardown - Construction Cost

\textbf{Out of futures contracts and swap contracts - which requires an initiation payment at the start?}

Neither. Neither futures contracts nor swap contracts require any initial investment, which makes this an unlikely reason to suggest swaps are preferred over futures contracts.

\textbf{Out of futures contracts and swap contracts - which is more bespoke?}

Swap contracts are more bespoke

\textbf{What types of commercial property is most likely to be affected by availability of single-family housing in the area?}

Multi-family. Single-family housing competes directly with multi-family housing for tenants.


\begin{framed}
	
\textbf{How would you classify a mutual fund manager of enhanced index ETFs who uses futures contracts - instead of holding the actual assets? Hedger, Trader or Speculator}

Trader - they have an interest in the market (as opposed to a speculator) but they are not hedgers. 

\textbf{Which of the following approaches to valuation is least likely to suffer from an inaccurate estimate of the capitalization rate? 1) Net asset value (NAV) using appraisals. 2) Market (relative) valuation using P/AFFO. 3) Discounted cash flow (DCF) valuation using NOI.}

P/AFFO


\textbf{When would you use 1) market value, 2) intrinsic value and 3) investment value? *****************}

1) Market value = for most real estate transactions // the typical buyer // analysts and banks prefer
2) Intrinsic value = takes into account the characteristics of property
3) Investment value = value specific to one investor

\textbf{Projects A, B, and C have costs of \$5 million, \$7 million, and \$10 million, respectively. Additionally, projects A, B, and C have implied land values of \$1.5 million, \1.75 million, and \$1.85 million. Which project is the highest and best use of the property?}

1) Ignore the costs - all negative
2) Choose C because it has the highest land value

\textbf{Which is more likely to benefit from corporations tax - a REIT or a REOC?}

A REIT.

\textbf{What is the biggest factor that affects trading for an index includes commodities traded on futures markets in the United States, Canada, and Japan, and the commodities are selection based. The performance of the index most likely will depend on the:}

Depend on the currencies - not the underlying commofities

\textbf{Compare Private Equity and Venture Capital in terms of 1) failures, 2) return rates?}

PE - few failures, moderate returns
VC - significant failures, high returns

\textbf{If a business is in contango / backwardation what can we say about the basis?}

positive / negative

\textbf{What is another name for insurance theory?}

Normal backwardation

\textbf{Company A has a unique and commercially viable product, however, it has not yet come to market. Which method of reaching a value conclusion will a provider of equity funds most likely apply when attempting to purchase Company A? ************}

Negotiation

Intrinsic value alone will be unlikely because there are no cash flows. Relative market valuation will be unlikely because the company has a unique product that will not compare reliably enough to find a benchmark. Negotiation will likely involve more than one baseline or top dollar estimate of value and is the most likely option.


\textbf{With respect to aligning the interests of managers and owners, which form of ownership would do a better job?}

Private Ownership

Private ownership is the best way to align the interests of managers and owners. Private firms have greater flexibility than public firms and, more important, do not ask managers to focus on the short-term performance of company stock, which can improve their capital allocation decisions.

\textbf{What are these high correlated with:1) natural gas, 2) industrial metals and 3) coffee?}

1) transport and electricity
2) construction - both industrial and residential
3) 

\textbf{Which of the following is most likely to be considered commercial property? 1) feed lot, 2) four-family flat, 3) 60-acre tract of pine tree}

1) feed lot = farmland
2  four-family flat = commercial
3) 60-acre tract of pine tree = farmland

\textbf{How many transactions are required for a Hedonic Index, Appraisal Index and a Repeat Transactions Index?}
Hedonic - One transactgion
Appraisal - None
Repeat Transactions - Many

\textbf{What is tenant concentration?}

It occurs when most of the lease is given to just one party - creating unneccesary risk

\textbf{If a company has to exit - what exit mechanisms tends to be approved by all parties? Management, investors and the markets? **********}

An IPO

\textbf{If a company doesn't need to exit - private status or public status? **************}

Private status. It is easier for a private company to align the interests of the owners and management.

\textbf{Gordon Growth Model for NOI}

GGM for NOI = (NOI) / rr - g = NOI / caprate

Note it is not,

GGM for NOI = [(NOI)(1+g)] / rr - g = NOI / caprate

\textbf{What is the estimated rental value (ERV)?}

It is a forecasted NOI

\textbf{how do you value a firm when there is an ARY and there is also DF? **************}

1) Value the PV from NOI and DF
2) Value the equity (ERV) from ARY,  at t = T
3) Value ERV at t=0, using PV and DF
4) Value = PV + ERV

\textbf{In a two-stage model; when you use the DDM at YR 4 - to what year should you add the solution? *******}

Add it to the year 3 solution

\textbf{What are the primary factors of single-family housing and multi-family housing?}

Demographic factors

\textbf{if the growth rate in capital and the growth rate in NOI are expected to diverge, which valuation method should be used?}

DCF

\textbf{What is 'straight-through-consumption'?}

Something that must be consumed immediately after extraction. For example natural gas or organ transplants.

\textbf{how does jet-fuel compare to normal fuel?}

1) Jet-fuel has a shorter shelf life
2) Storage costs are much lower for jet fuel

\end{document}