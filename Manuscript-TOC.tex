\documentclass[7pt]{article}
\usepackage[framemethod=TikZ]{mdframed}[2013/07/01]
\usepackage[T1]{fontenc}
\usepackage{framed}
\usepackage{physics}
\usepackage{amssymb}
\usepackage{changepage}
\newlength{\rulethickness}
\setlength{\rulethickness}{1.2pt}
\newlength{\rulelength}
\setlength{\rulelength}{15cm}
\usepackage[left=2cm, right=2cm, top=2cm]{geometry}
\usepackage[utf8]{inputenc}
\usepackage{setspace}
\usepackage{amsmath,amsthm,amscd}
\usepackage[colorlinks]{hyperref}
\usepackage{setspace}
\usepackage{xcolor}
\usepackage{xparse}
\usepackage{soul}

\linespread{1.25}
\NewDocumentCommand{\inn}{mo}{%
	\langle #1\rangle
	\IfValueT{#2}{^{}_{\mspace{-3mu}#2}}%
}
\DeclareMathAlphabet{\mathpzc}{OT1}{pzc}{m}{it}

\begin{document}

\begin{enumerate}
%%Part 1: \\
\item	 Introduction \\
\item	Conventional Machine Intelligence \\
\item   	 Likelihood Estimators, Classifiers and Discriminators
 \\
\item    	Backtesting Simple Machine Learning Algorithms (K-means and Clustering)
\\
\item  	Building Machine Learning Classifiers 
\\
\item     	Time Series (Training data and Testing)
\\
\item   	Linear Classifiers
\\
\item    	Duality, Lagrange Duality and Constrained Optimizers
\\
\item   	SVM 
\\
\item  	Optimal Regressors, Regression, Non-Parametric Regression, OLS and Kernel Methods
\\
\item   	Unsupervised Learning
\\
\item	Reductions in Dimensionality, ICA and PCA
\\
\item	Advanced Topics
\vspace{0.25in}
\\Part 2:
\item 	Introduction to design and implementation in a parallel programming environment
\\
\item 	Contemporary GPU architecture
\\
\item  	Parallel Computing on GPUs
\\
\item 	Parallel simulation, libraries and software frameworks (dynamics, parallel  patterns, atomics, streams and thread cooperation). 
\\
\item 	Conclusion
\end{enumerate}

\newpage

Science is always presupposing some basic concepts must be held to be useful. These absolute presuppositions (Collingwood), tend to rarely debated and form the framework for what have been termed “paradigm” by Kuhn. Our currently accepted scientific model is predicated on a set of presuppositions that have difficulty accommodating holistic structures and relationships otherwise not geared towards incorporating non-local correlations

%% refer to Algorithmic Information Theory and Cellular Automata Dynamics by Cervelle and Durand.
\section*{Next}
The goal of this book is to develop machine learning approaches that scale to massively parallel implementations; thereby to test theories or otherwise implement models for the purposes of data evaluation. We consider machine learning as  methods that automatically detect relationships in data and then use these uncovered relationships for decision making or prediction when faced with randomness. The book is empirical in its approach. The book is targeted at later year undergraduate students and seasoned graduate students in electrical engineering, econometrics, computer sicence or physics. This books approaches machine learning from the perspective that best approach to addressing learning problems is to adopt an HPC approach (HPC, cloud or something else). Machine learning is related to electrical engineering and in turn statistics but differs in the assumptions and language.
\section*{Introduction}
One of the primary element in the development of information theory is norm. By generalization we may define the norm in the Euclidean space $R^N$,
\begin{equation}
||\gamma|| = [y^2_1 + y^2_2 + y^2_3 + .... + y^2_N]^{1/2}
\end{equation}
which defines the norm as basis to assess the "magnitude" of vectors through a generalization of the Euclidean distance. This therefore induces a notation for the magnitude of vectors in terms of an orthogonality, using a scalar product measure,
\begin{equation}
\inn{\gamma,\gamma}
<\gamma,\gamma> \expval{\gamma,\gamma} = [y^2_1 + y^2_2 + y^2_3 + .... + y^2_N]^{1/2}
\end{equation}
%% http://detexify.kirelabs.org/classify.html
%%logit model = traditional statistical classification approach
%%hierarchical structures = a machine learning view
\textcolor{red}{Empirical covariance matrices computed on series of observations from a random vector, therefore to estimate the linear comovement between the random variables that constitute the random vector.}

%%% Use Cover and Thomas to check the proof
\subsection*{\textcolor{purple}{Chapter 0: Random Variables}}
The abstraction of {random variables} is principally important within information theory and in modern physics. In information theory we tend to refer to data in terms of \emph{random variables};  we define here  {random variables} and provide also a few related concepts. We initially give a general mathematical definition based on a probability space $[ \Omega, \varepsilon, p ]$, where $[ \Omega, \varepsilon ]$ is the \emph{sample space}; note where $p[\Omega] = 1$ then $p$ is our $\emph{probability measure}$. The measurable space $\Omega$ and $p$ consist of $\varepsilon$  a $\sigma$-algebra of subsets called the \emph{events}. Therefore $\Omega$ is the space of outcomes, such that $\omega \in \Omega$ is an outcome, $Y = Y[\omega]$ is considered a random variable with an alphabet $\mathpzc{Y}$ and $K_Y$ is the range of random variables called the \emph{state space}.   We can consider random variable $Y$ as the assessed state of a state space $K_Y$ with a probability $p[y]$; where $Y$ is distributed uniformly as $p[y] = 1/||\mathpzc{Y}||,\forall y \in  \mathpzc{Y}$. We define that in the $\sigma$-algebra $\varepsilon$: $\varepsilon \neq 0$, $\Omega$ is the \emph{certain event}, $Y$ is an event in so much as its complement is an event $Y := \Omega / Y^{c}$, if $[y_i]_{i \in N}$ and if $[y_i]_{i \varepsilon N}$ is a family of events then $U_{i\in N} y_i$ and $\theta$ is the \emph{impossible event}.   Let us consider then the probability space $[\Omega, \varepsilon, p]$  and the measurable space $[\chi, \mathpzc{F}]$; the random variable $Y$ is a function from $\Omega$ to $\chi$ which is measurable with respect to the $\sigma$-algebras, with a range $[\chi,\mathpzc{F}]$.

%% read notes_statistics_1 and notes_statustucs_2

\section*{\textcolor{purple}{Chapter 1: Entropy}}
Entropy  is the expected value of surprise events; where the two variables fail to adhere to a normal bivariate distribution any inferences from the correlation metric will be minimal. Therefore in more rigid scenarios, information theory must play a hand.

We consider a discrete random variable or an event $Y$ from a finite sample space $K_{Y}$  and with a probability mass function $p[y]$, $y \in \mathpzc{Y}$,
\begin{equation}
p[y] = \dfrac{1}{\inn{a,a}^{1/2}}
\end{equation}
and where $i[y]$  is monotonic on the probability distribution $p[y]$,
\begin{equation}
i[y] = \dfrac{1}{p[y]}
\end{equation}
which is termed the \emph{surprisal} of $y$, i.e. a surprising observation is considered as the inverse of their probability, such that a particular instance with a low probability of occurring will have a high degree of surprise. We can assess the discrete random variable $Y$ and ask how much information is received by observing a specific instance for this variable, the \emph{degree of surprisal} from the receipt of $y$. Compared to scenarios with events considered to be highly likely or scenarios whose events will otherwise be known to occur with certainty, we consider $y$ to be a high unlikely event; compared to the earlier scenarios, this has a higher degree of surprisal and by-inference less information. [notes/propoerties] We therefore consider information as being measured where $p[x] \leq 0$, we must have in the case where we have two random variables independent of each other $y$ and $x$, $p[y].p[x] = p[y,x]$. We must have therefore a certain measure of information $h[y]$ that is monotonic on $p[y]$, but where we have two statistically independent and discrete random variables $y$ and $x$, the measure  of information $h[y,x]$ obtained through the observation of the independent events is being given by $h[y] + h[x]$. One consequence of the three properties is  we must have therefore a measure,
\begin{equation}
h[y] = \ln [ 1 / p[y] ]\longleftrightarrow H[Y] =  \sum_{y \in \mathpzc{Y}} \ln [ 1 / p[y] ]
\end{equation}
where $H[Y]$ is termed the \emph{entropy} of $Y$ ie  the mean amount of information obtained of $Y$ and the expectation of $h[y]$ on a probability mass function $p[y]$;
\begin{equation}
\begin{split}
H[Y] & =  \sum_{y \in \mathpzc{Y}} p[y] \ln [1/p[y]] \\
H[Y] & = - \sum_{y \in \mathpzc{Y}} p[y] \ln [p[y]] \text{ nats}
\end{split}
\end{equation}
where entropy is  log[.] of an information surprise multiplied by the its proportion of observations in a set of events. An unexpected message or an message that is expected but that has been otherwise coded at the source so as to appear random may be therefore said to have information content or  high entropy. Entropy may be considered as evaluating the amount of uncertainty associated with a random variable; note therefore an expected event under an entropy model contain no information.
%%The logarithm may be in any base but is usually set at base 2, wherein the unit of entropy is termed the bit. %% \footnote{If we were to denote the logarithm in base c, the entropy denoted as $H_c(X)$, may be found from other entropies as $H_c(x) = log_c^b H_b(x)$}
We must infer that maximum entropy for a discrete random variable  corresponds to a normal distributed random variable, i.e. equal distributions across states of the variable; entropy therefore has its greatest value at $\ln [|| \mathpzc{Y}||]$ when $Y$ is uniformly distributed variable,
$$
H(y) = \sum_{y \in \mathpzc{Y}}  p(y) \ln [1/||\mathpzc{Y}||], \forall y \in \mathpzc{Y}
$$ %check this both lines
where we note that $H[X] \leq \log|\mathpzc{Y}|$. Therefore given the probability mass function $p[y]$ of $Y$ and  $u[y] = 1/|y|$ as the uniform probability mass function for $Y$ then,
$$
D[p||u] = \sum p[y] \ln \dfrac{p[y]}{u[y]} \\
=\log |\mathpzc{Y} | - H[y]
$$
Since $D[p||u] > 0$, we observe that entropy is upper bounded by the logarithm of the alphabet size $|\mathpzc{Y}|$,  $H[y] \leq \log |\mathpzc{Y}|$. Note, here that we have adopted the convention based off continuity arguments that
\begin{equation}
\begin{split}
\lim\limits_{ p \rightarrow 0^{+} } p \ln[ p]  & = 0
\end{split}
\end{equation}
therefore we may generalise based off continuity arguments that $p(y=0) \ln [ p[y=0]] = 0$, $0 \ln [0/u] = 0$, $p \ln [p/0] = 0$, $0 \leq p[y] \leq 1$  {and} $- \log p[y] \geq 0$. Therefore when we consider the joint event with a  discrete random variable $Y$ with an outcome $y$ whose probability mass function is $p[y]$ from an alphabet $\mathpzc{Y}$ and discrete random variable $X$ with an outcome $X$ whose probability mass function is $p[x]$ from an alphabet $\mathpzc{X}$ we obtain the \emph{differential entropy},
%%**(ditto: correct and reference)
\begin{equation}
\begin{split}
H[Y,X] & = - \sum_{y \in \mathpzc{Y}} \sum_{x \in \mathpzc{X}}  p[y,x] \ln[ p[y,x]] \\
H[Y,X] & = - \sum_{y,x \in \mathpzc{Y} \times \mathpzc{X}} p[y,x] \ln[ p[y,x]]
\end{split}
\end{equation}
%%**(ditto: correct and reference)
wherein we find, $H[Y,Y] = H[Y], H[X,X] = H[X], H[Y,X] = H[X,Y], H[Y,X] \geq \max [ H[Y], H[X] ]$ and $ H[Y,X] \geq H[Y] + H[X]$.
\subsection*{\textcolor{purple}{The Differential Entropy}}
In continuous time, consider the random variable $Y$ which may be considered over discrete time periods as having a probability mass function  $p[Y]$,
%%Add figure 8.1 from thomas and corver
We bucket the range of $Y$ into lengths of $\Delta$ and assume that the probability mass function $p[y]$  remains continuous within each of these buckets. Therefore from the mean value theorem there must exist within each bucket $y_i$ can be defined as,
%mean value theorem of integral calculus) an interval of \Delta the quantised value of x_ holds as,
\begin{equation}
\int_{i\Delta}^{(i+1)\Delta}  dx p[y]  = p[y_i] \Delta
\end{equation}
where the  probability mass function over an internal $\Delta$ is given as $p[y_i]\Delta$ for variable $y_i$; therefore we may define the  quantized random variable as given by,
\begin{equation}
Y_{\Delta} = y_i \text{     }\forall Y,i\Delta \leq Y \leq (i+1) \Delta
\end{equation}
therefore the entropy after bucketing the range of $Y$ is therefore given by,
%% checked
\begin{equation}
H_{\Delta} = - \sum_{i=1}^{N}  p[y_i] \ln p[y_i] \\
= - \sum_{i=1}^{N}  p [ y_i ] \Delta [ \ln p[ y_i ] \Delta ] \\
= - \sum_{i=1}^{N}  \Delta  p [ y_i ]  \ln p [ y_i ] - \ln \Delta
\end{equation}
Where we consider the constraint $\int p[y]  dy = \sum p[y] \Delta = 1$ to hold. When $p[y]\ln[f[y]]$ is Riemann integrable; $- \sum_{i=1}^{N}  \Delta  p [ y_i ]  \ln p [ y_i ]$ approaches the integral of $-p[y]\ln p=[y]$ as $\Delta \rightarrow 0$ given its Riemann integrability.
%% Discuss the differential entropy
%%Discuss the differential entropy
We may therefore obtain the entropy of a discrete random variable from the differential entropy of the continuous random variable as,
\begin{equation}
\begin{split}
\lim\limits_{\Delta \rightarrow 0} H_{\Delta} + \ln \Delta  \rightarrow h[\Delta] = - \int p[y] \ln p[y] dy
\end{split}
\end{equation}
which discretised and with a probability mass function defined over multiple continuous variables provides us with the \emph{differential entropy} estimator,
$$
H[Y] = - \int p[y] \ln p[y] dy
$$
We note that entropy of the continuous random variables  differs from the entropy of the quantized  continuous random variable by a $\ln \Delta$; hence the entropy of the n-bit quantisation of the continuous random variable may be given by $n + h[Y]$ bits

%%suffice to say that to quantise the continuous random variable a large number of bits are required

\subsection*{\textcolor{purple}{Maximum Entropy for A Continuous Random Variable}}

Similar to the discrete random variable, we wish to obtain an estimator to the maximisation of the continuous random variable through a constrained maximisation,
$$
H[Y] +\lambda_1 \Biggr(  \int_{-\infty}^{+\infty} (dy [p[y] - 1]) \Biggl) + \lambda_2 \Biggr(  \int_{-\infty}^{+\infty} dy [y p[y]  - \mu] \Biggl) + \lambda_3  \int_{-\infty}^{+\infty} \Biggr( dy [ y - \mu]^2 p[y]  - \sigma^2 \Biggl)
$$
Where the distribution that maximises entropy given a continuous random variable is then found to be Gaussian random distribution
\begin{equation}
%%p[y] = \dfrac{1}{(2 \pi \sigma^2)^{0.5}} \exp \biggr{ - \dfrac{(y - \mu)^2}{2 \sigma^2} \biggr}
\end{equation}
Therefore the maximum entropy  the continuous  variable is given by,
\begin{equation}
\begin{split}
h[p[y]] & = - \int p[x] [ -1/2 ([y - \mu] \sigma^{-2}  [y - \mu] - \ln [\sqrt(2 \pi) |\sigma^2|^{1/2}] \\
h[p[y]] & = \dfrac{1}{2} E [ \sum_{i,j} [y_i - \mu_i] \sigma^{-2}[y_j- \mu_j]] +\dfrac{1}{2} \ln[2 \pi] |\sigma^2| \\
h[p[y]] & = \dfrac{1}{2} \sum_{i,j} E [  [y_i - \mu_i] [x_j- \mu_j]]\sigma^{-2}  +\dfrac{1}{2} \ln(2 \pi) |\sigma^2| \\
h[p[y]] &= \dfrac{1}{2} \sum \sum K_{ij} [K^{-1}]_{ji} + \dfrac{1}{2} \ln [2 \pi] | \sigma^2 | \\
h[p[y]] & = \dfrac{1}{2} \sum [ K_{ij} [K^{-1}]_{ji} ] + \dfrac{1}{2} \ln [2 \pi] | \sigma^2 | \\
h[p[y]] &= \dfrac{1}{2} \sum I + \dfrac{1}{2} \ln [2 \pi] | \sigma^2 | \text{nats} \\
h[p[y]] &= \dfrac{1}{2} + \dfrac{1}{2} \log [2 \pi] | \sigma^2 | \text{bits}
\end{split}
\end{equation}
Where changing the base then gives,
$$
h[p[y]] = \dfrac{1}{2} \log [2 \pi e] | \sigma^2 |
$$
Therefore we may generalise the average number of bits required to describe a continuous variable $Y$ whose outcome is $y$ to n-bit accuracy as,
$$
h[p[y]]  + n  =  \dfrac{1}{2} \log [2 \pi e] | \sigma^2 | \text{bits}
$$
\subsection*{\textcolor{purple}{Conditional Entropy}}
We may now obtain the derivation of the entropy of $Y$ conditioned on $X$; where if we draw pairs of values from a joint density function $p[y,x]$ the supplemental information required to identify $Y$ is given on average as,

$$
H[Y|X] = - \int p[y,x] \log p[y|x] dy dx
$$
in general $ p[y|x] = p[y,x] / p[x]$ we obtain,
%%maybe expand the proof%%
$$
H[Y|X] = H[Y,X] - H[X] = - \sum_{x \in K_x} p[x] \sum_{y \in K_y} p[y|X ]\log p[x|Y]
$$
Our definition provides the insight $H[Y|X]$ as the unknown supplemental information in Y when we privy to X; where $H[Y] \geq H[X|Y], H[Y|Y] = 0, H[Y,X]$ is the differential entropy of a joint distribution $p[y,x], H[Y]$ is the marginal distribution $p[y]$.
\subsection*{\textcolor{purple}{Cross Entropy}}
We may obtain also the \emph{cross entropy} which allows us to infer the information unknown in $Y$, when we use an approximating probability mass distribution $u[y]$ to approximate the posterior probability mass distribution $p[y]$; or the converse, the information gained in updating $p[y]$ to update $u[y]$ \footnote{$u[y]$ and $p[y]$ are discrete probability densities  defined on the same state space; this differs from a differential entropy where $u[y]$ and $p[y]$ were not assumed to be from the same state space.},
$$
H_{\chi} [p[y]||u[y]] = H[y] + D_{KL} [p[y]||u[y]] = - \int p(y) u(d) dy \longleftrightarrow  H_{\chi} [p[y]||u[y]] - H[y] = D_{KL} [p[y]||u[y]]
$$
where $H_{\chi}$ is the cross entropy and $D_{KL} [p[y]||u[y]]$ is known as the Kullback-Liebler divergence or KL divergence between $p[y]$ and $u[y]$.
%%The cross entropy is the uncertainty in an observation of $y$ using the suboptimal dist5ribtuion $y[y]$.
%In any instance where we have a distribution $u[y]$  , opy; we may  then construct a forward error correction %scheme or space-time code based on u(y) to thereby transmit y to a receiver.
\subsection*{\textcolor{purple}{KL Divergence}}
%The KL to $u[y]$  the prior  to therefore achieve $p[y|x]$ the posteriori.

We can also obtain the \emph{relative entropy} or \emph{KL divergence} on the \emph{a prior} and the \emph{a posteriori} by the likelihood model specified by\footnote{unlike in the differential entropy where both $u[y]$ and $p[y]$ are considered to be independent distributions,in the KL divergence both $u[y]$ and $p[y]$ are defined on the same probability state space}. The relative entropy is the "distance" between two distributions or the additional bits required when we assume our observations are an accurate approximation of a known distribution \newline \footnote{Were we to encode information $X$ given a probability $p[x]$, using an assumed distribution $q[x]$ we would have to have codewords proportional to $\log 1/p[x]$. An expected codeword length would be, $E[X] =  \sum_{x} p[x] \log \bigr( 1/q[x] \bigl)=   \sum_{x} p[x] \bigr(1 + \log [ 1/q[x] ] \bigr) = H[x] + 1$. Now to use the wrong distribution for $q[x]$, $E[X] =  \sum_{x} p[x] \log \bigr( 1/q[x] \bigl)=   \sum_{x} p[x] \bigr(1 + \log [ 1/q[x] ] \bigr) = \sum_{x} p[x] +  \sum_{x} p[x] \log [ 1/q[x] ]  = \sum_{x} p[x] +  \sum_{x} p[x] \log [ p[x]/[p[x]q[x]] ] = \sum_{x} p[x] +  \sum_{x} p[x] \log [ q[x]/p[x] ]  +  \sum_{x} p[x] \log [ 1/p[x] ] = 1+ D_{KL} [p[y]||u[y] +  H[x]$. The penalty in the posterior for incorrectly using the wrong distribution will therefore be the KL.   }
$$
D_{KL} [p[y]||u[y]] = - \int p[y] \ln u [y] dy - \biggl(  - \int p[y] \log p[y] dxy \biggr)
= -  \int p[x] \ln \dfrac{u[y]}{p[y]} dy
=  \int p[x] \ln \dfrac{p[y]}{u[y]} dy
$$
furthermore, maximising the likelihood model by observing $y$ under a model parameterized by $u[y]$ of a adequately assumed form (e.g. Gaussian or factored formed) is  equivalent to minimising a KL distance between the received codeword sent through a channel with a distribution $p[y]$ and the codeword at the source with a marginal $u[y]$; this is termed as the cross-entropy.
$$
D_{KL} [p[y]||u[y]] = H_{\chi} [p[y]||u[y]] - H[x] \longleftrightarrow H_{\chi} [p[y]||u[y]] = H[x] + D_{KL} [p[y]||u[y]] = - \int p[x] u[d]h dx
$$
when two variables are found to be from a joint distribution, we may determine whether the two variables are diverging and therefore are tending towards being independent by the considering KL divergence between the joint distribution $p[y,x]$ and the marginal distributions $p[y]p[x]$.
%%Princeton notes%%

$$
D_{KL} [p[y,x]||[p[y][p[x]] =  - \int \int dx dy p[y,x] \log \dfrac{ p[y]p[x] }{p[y,x] }
$$
where the KL divergence is considered an update in $x$ achieved by observing $y$, the surprisal. The KL divergence may also therefore thought of as the 'information gain' achievable in $x$ by observing $y$. So that $-D_{KL} [p[x]||q[x]] \leq 0$;
\begin{equation}
\begin{split}
-D_{KL} [p[y]||q[y]] & = - \sum_{y\in\mathpzc{Y}} p[y] \log\dfrac{p[y] }{q[y] } = \sum_{y\in\mathpzc{Y}} p[y]  \log\dfrac{q[y]}{p[y]} \\
& \leq \log \sum_{y\in\mathpzc{Y}} p[y] \dfrac{q[y]}{p[y]} \\
& \leq \log \sum_{y\in\mathpzc{Y}} {q[y]} = \log 1 = 0  \\
\end{split}
\end{equation}
as $\log[.]$ is a concave function\footnote{We use Jensen's inequality; where $f$ is a concave function, $$ f \Bigr( \sum_{i=1}^N \lambda_i y_i \Bigr) \leq \sum_{i=1}^N \lambda_i f(y_i)$$ }; hence, we may infer then, $D_{KL} [p[y]||q[y]] = 0$ if and only if $p[y] = q[y]$. Another product of this observation is that given no-knowledge of $Y$, maximum entropy is achieved where the prior distribution $q[y]$ is null; let us set $q[y] = u[y] = 1/|\mathpzc{Y}|$,
\begin{equation}
\begin{split}
0 \leq  D_{KL} [u[y]||p[y]] & = \sum_{y\in\mathpzc{Y}} {p[y]} \log\dfrac{p[y] }{u[y] } = \sum_{y\in\mathpzc{Y}} {p[y]} [\log p[y]] - \sum_{y\in\mathpzc{Y}} {p[y]} (\log u[y])\\
& = -H[Y] + log[\mathpzc{Y}]
\end{split}
\end{equation}
this corresponds to $u[y]$ being  \emph{uniformly distributed}; maximum entropy is achieved where the least assumptions are made about an initial distribution. The KL divergence measures a "distance" between distributions however it is not a Euclidean distance measure and it is not a metric; it is not symmetric ($D_{KL} [p[y]||q[y]] \not\equiv D_{KL} [q[y]||p[y]]$) and does not satisfy the triangular inequality ($D_{KL} [p[y]||q[y]] \not\leq D_{KL} [p[y]||u[y]] + D_{KL} [u[y]||q[y]]  $)\footnote{The $D_{KL} [p[y]||q[y]]$ and $D_{KL} [q[y]||p[y]]$ divergences will only be equal were $q[y]$ were sufficiently complex}. However, an important property of $D_{KL} [p[y]||u[y]]$ is its  a non-negative measure;  $D_{KL} [p[y]||u[y]] \geq 0$ and $D_{KL} [p[y]||q[y]] = 0$ iff $p[y] = q[y]$.
%\textcolor{purple}{Heuristically, the probability density function on {𝑥1,𝑥2,..,.𝑥𝑛} with maximum entropy turns out to be the one that %orresponds to the least amount of knowledge of {𝑥1,𝑥2,..,.𝑥𝑛}, in other words the Uniform distribution.}
\subsection{Forward-divergence and Reverse-divergence}

As we previously discussed the KL divergence measure is not symmetric, i.e. $D_{KL} [p[y]||q[y]] \not\equiv D_{KL} [q[y]||p[y]]$. This therefore provides two different objective functions when attempting to approximate $p[y]$:
\begin{enumerate}
	\item A \emph{forward KL-divergence}, $\arg \min D_{KL} [p[y]||q[y]] \longleftrightarrow \arg \max_{\theta} \Bigl[ \lim\limits_{ N \rightarrow \infty } \dfrac{1}{N} \sum^N_{i=1} \log q[y_i|\theta]\Bigr]  $
	\item A \emph{reverse KL-divergence}, $\arg \min D_{KL} [q[y]||p[y]] \longleftrightarrow  \arg \max_{\theta} \Bigl[ \lim\limits_{ N \rightarrow \infty } \dfrac{1}{N} \sum^N_{i=1} \log p[y_i]\Bigr]  + H[q[Y]] \geq H[q[Y]]\\
	$
\end{enumerate}
where $p[y]$ is the known distribution, minimizing forward KL-divergence $D_{KL} [p^{source}[Y]||[q[Y|\theta]]$ with-respect-to parameters $\theta$ is equivalent to the \emph{maximum likelihood estimator} (MLE)\footnote{The basic Maximum Likelihood Estimator (MLE) (Fisher1922) searches for the parameter $\theta$ that likely generated $Y$; note that MLE may be biased and may likely not be an optimal minimum variance estimator},
\begin{equation}
\begin{split}
& \arg \min_{\theta} D_{KL} [p^{source}[Y]||[q[Y|\theta]] = \sum_{i} p^{source}[y_i] \log \dfrac{q[y_i|\theta]}{p^{source}[y_i]}  \\
& =  \arg \min_{\theta} \Bigr[  \sum_{y} p^{source}[Y] \log {p^{source}[Y] }   - \sum_{y} p^{source}[Y] \log {q[Y|\theta]} \Bigr]\\
& =  -H[p[Y]] - \arg \min_{\theta} \Biggr[  \Bigl[ \lim\limits_{ N \rightarrow \infty } \dfrac{1}{N} \sum^N_{i=1} \log q[y_i|\theta]\Bigr] \Biggr]  \\
&  = -H[Y] + \arg \min_{\theta} \Bigl[ - \lim\limits_{ N \rightarrow \infty } \dfrac{1}{N} \sum^N_{i=1} \log q[y_i|\theta]\Bigr]  \\
&  =  \arg \max_{\theta} \Bigl[ \lim\limits_{ N \rightarrow \infty } \dfrac{1}{N} \sum^N_{i=1} \log q[y_i|\theta]\Bigr]  \\
%%&  \propto  \sum^N_{i=1} \log p[x_i|\theta]
\end{split}
\end{equation}
we know that no other consistent estimator will provide a lower asymptotic mean and therefore the MLE is asymptotically efficient; therefore this allows us to adopt a forward KL in Machine Learning. We observe that $-H[Y]$ is effectively constant and disappears from the expression, thereby limiting $[[q[Y|\theta]$  in optimally tracking the modes of $p^{source}[Y]$.  The intuition  from the forward KL-divergence that it's objective is that wherever samples of $p^{source}[Y]$ are optimal, $[q[Y|\theta]$ must also be optimal; this leads to maximization of a globally optimal distribution of $q[Y]$ that attempts to match all $p^{source}[Y]$ modes. However for the reverse KL-divergence $D_{KL} [[q[Y|\theta]||p^{source}[Y]]$ where $p^{source}[Y]$ is known to be distinctly multimodal and a Gaussian $q[Y|\theta]$ is employed, the KL-divergence will tend to match local modes. Thereby a minimization of $\min D_{KL} [q[y]||p[y]]$ constrains $q[y|\theta]$ to assume the properties of $p[y]$,
%%\textbf{note_diagram_1}
\begin{equation}
\begin{split}
\arg \min D_{KL} [q[y|\theta]||p[y]] & = \sum_{y} q[Y|\theta] \log \dfrac{q[Y|\theta]}{p^{source}[Y] }  \\
& =  \arg \min_{\theta} \Bigr[ \sum_{y}  q[Y|\theta]  \log {q[Y|\theta]} \Bigr] + \arg \min_{\theta} \Bigr[ - \sum_{y} p^{source}[Y] \log {p^{source}[Y] } \Bigr] \\
& = -H[q[Y]] + \arg \min_{\theta} \Bigl[ - \lim\limits_{ N \rightarrow \infty } \dfrac{1}{N} \sum^N_{i=1} \log p[y_i]\Bigr]  \\
& = \arg \max_{\theta} \Bigl[ \lim\limits_{ N \rightarrow \infty } \dfrac{1}{N} \sum^N_{i=1} \log p[y_i]\Bigr]  + H[q[Y]]  \\
\end{split}
\end{equation}
we infer therefore an objective such that wherever $[q[Y|\theta]$ is known to be optimal samples of $p^{source}[Y]$ must also be optimal; this leads to a maximization; sampled $p^{source}[Y]$ modes are matched against locally optimal $q[Y\theta]$, with $H[q[Y]]$ ensuring that the reverse KL-divergence is not overly narrow. As $H[q[Y]]$ remains in the objective, the modes of $p^{source}[Y]$ may be optimally tracked. We note that $D_{KL} [q[y|\theta]||p[y]]$ is infinite where $[q[y|\theta] > 0$ and $p[y]=0$; therefore we term the reverse KL-divergence to be \emph{zero forcing} as we prefer $q[y]=0$ when $p[y]=0$, therefore typically underestimating $p[y]$.  An reverse KL-divergence is unable match all nodes and therefore tends to be better suited to computing divergence, particularly where  $p^{source}[Y]$ is known but multimodal or considered to be intractable.
%%\textbf{note_diagram_1}
\subsection{Jesen-Shannon divergence and Renyi divergence}
\subsection*{\textcolor{purple}{Mutual Information}}

\begin{equation}
\begin{split}
D_{KL} [p[y,x]||[p[y][p[x]] = I[Y,X]
\end{split}
\end{equation}
we in infer that $I[y,x]$  the \emph{mutual information} of a random variable, is the reduction in the uncertainty in $y$ by an updated observation of $x$ as;
\begin{equation}
\begin{split}
I[Y,X] & = H[Y] - H[Y|X] \\
I[Y,X] & = - \sum_{y \in K_Y} p[y] \log p[y] - \sum_{x \in K_X,y \in K_Y } p[y,x] \log p[y|x] \\
I[Y,X] & = - \sum_{x \in K_X,y \in K_Y } p[y,x] \log p[y] - \sum_{x \in K_X,y \in K_Y } p[y,x] \log p[y|x] \\
%I[y,x] & = - \sum_{x \in K_X,y \in K_Y } p[y,x] \log \dfrac{ p[y,x]}{p[y]p[x]} \\%
I[Y,X] & = - \sum_{x \in K_X,y \in K_Y } p[y,x] \log \dfrac{ p[y|x]}{p[y]} \\
\end{split}
\end{equation}
where the mutual information  by virtue of the observation of $X$ we obtain a prior $p[y]$  and a posteriori $p[y|x]$ that reduce the uncertainty. When we observe $Y$ to be independent of $X$ Note also $I[Y,X] = I[X,Y] \geq 0$ provided $X$ and $Y$ are independent. We may then infer $I[Y,X] =  H[X] - H[X|Y]$, and $I[Y,Y] = H[Y] - H[Y|Y] = H[Y]$.  We note that $I[Y,X] =  H[Y] + H[X] - H[Y,X]$ as $H[Y,X] = H[Y] + H[X|Y]$.

\section{Chapter X: Optimal Divergence}

\subsection{Variational Autoencoder}

The penalty  to describe $[X]$ when using the wrong distribution $q[x]$ describes $[X]$ was shown to be $D_{KL} [p[y]||u[y]$; such that the length of the codeward $E[L[X]] = 1+ D_{KL} [p[y]||u[y] +  H[x]$. \emph{Generative models} build models of distributions, which attempt to capture the materiality of the true-distribution $p[x]$ , as accurately  possible. For instance Generative Adversarial Networks (GANs), a class of non-parametric non-linear estimators, learn the structure of latent dependencies in a high-dimensional space and then attempt to generate a model in a low-dimension by minimizing $D_{KL} [p[y]||u[y]$; this may be shown is being equivalent to maximizing the log-likelihood ratio by optimizing model parameters $\theta$ when observing $Y$,
\begin{equation}
\begin{split}
& D_{KL} [p^{source}[Y]||[q[Y|\theta]] = \sum_{y} p^{source}[Y] \log \dfrac{p^{source}[Y]}{q[Y|\theta]}  \\
& =   \sum_{y} p^{source}[Y] \log {p^{source}[Y] } - \sum_{y} p^{source}[Y] \log {q[Y|\theta]}\\
& =   - \Bigl[ \lim\limits_{ N \rightarrow \infty } \dfrac{1}{N} \sum^N_{i=1} \log p[x_i|\theta]\Bigr]+ H[Y]  \longleftrightarrow \text{MLE}_{\hat{\theta}} = \arg \min_{\theta} D_{KL} [p^{source}[Y]||[q[Y|\theta]] = \arg \max_{\theta} \Bigl[ \lim\limits_{ N \rightarrow \infty } \dfrac{1}{N} \sum^N_{i=1} \log p[x_i|\theta]\Bigr]  \\
%%&  \propto  \sum^N_{i=1} \log p[x_i|\theta]
\end{split}
\end{equation}
we infer therefore that maximizing the maximum-likelihood (LHS) is equivalent to a minimization of the RHS,
\begin{equation}
\begin{split}
D_{KL} [p^{source}[Y]||[q[Y|\theta]] &  \propto \dfrac{1}{N} \sum^N_{i=1} \log p[x_i|\theta]
\end{split}
\end{equation}
\subsection{Variational Bayesian Methods}

\section{Chapter 0: Advanced Topics}
An emerging field in Machine Learning is the adoption of techniques from quantum computing to aid the development of ML algorithms.
Although the concept of probability distributions does not hold in quantum theory, a quantum-mechanical amplitude probability is used to describe the corresponding random variable.
m
\subsection{Quantum Formalism}
%%A brief introduction to quantum formalism
The basic notion of \emph{standard formalism} in quantum mechanics, considers \emph{observables} as being best represented by a Hermitian Matrix or a Hermitian operator. The Hermitian exists in a complex linear space\footnote{Let us consider a matrix $Z = [z_{i,j}]$; iff $[z_{i,j}
	=  [\overline{z}_{j,i}][,\forall z \in \mathbb{C}$, then $Z$ is the Hermitian}, a Hilbert space $H$, which is the complete space defined by the scalar product $\inn{\Psi_1,\Psi_2}$, which gives rise to the  norm $||\Psi|| = \inn{\Psi,\Psi}$ and distance $d_{H}[\Psi_1,\Psi_2] = ||\Psi_1 - \Psi_2 ||$\footnote{We can define the scalar product as the function which maps a Cartesian product of the Hilbert space $H \times H$ to a complex number $\mathbb{C}$.}. The scalar product confers:
\begin{enumerate}
	\item $\inn{\Psi | \Psi} \geq 0$, \\
	\item $\inn{\Psi_1,\Psi_2} = \overline{\inn{\Psi_1,\Psi_2}}$, \\
	\item $\inn{\phi|k_1 \Psi_1 + k_2 \Psi_2} = k_1\inn{\phi|\Psi_1} + k_2\inn{\phi|\Psi_2} {        }\forall k_1,k_2 \in \mathbb{C}$
\end{enumerate}
where $H$ consists of Project Valued Measures; such that given a projection $\pi_b$, the identity operator $\pi_b^2 = \pi_b$, $\pi_b \geq 0$ and $\sum_{b} \pi_b = I$  an orthogonal decomposition must be given by $\text{tr}[ \pi_b \pi_{b'} ] = 0$ where the subspaces span the state space, $\sum_{b=1}^{k} \pi_b = U$.
\subsection{Pure states**}
Let us consider a normalization of vectors in the complex Hilbert Space $H$;
$
\braket{\Psi} = 1
$
this vector $\ket{\Phi}$ of unit length importantly is termed the \emph{pure state} of a quantum system; such that there for pure states, $\Psi_a \sim \Psi_b$ given a phase factor $e^{i \theta}$ iff,
$$
\Psi_a \equiv e^{i\theta}\Psi_b \: \: \forall \theta \: \in [0,2 \pi)
$$
Therefore we must have an orthonormal projection operator on a normalized vector of each pure state $\Psi$ as\footnote{Note, the projection operator corresponding to a pure state is expressed by the matrix $P_{\psi} = {p}_{j,i}$, it has the defining properties 1. $\hat{P}_{\Psi}^2 = \hat{P}_{\Psi}$, 2. tr[$\ket{\Psi}\bra{\Psi}$]=1, \st{$ \text{tr} P = \sum_j {P}_{i,i}$}  3. ${P}_{i,j} = \overline{{P}}_{j,i}$ and 4. $\inn{P\phi,\phi} \geq 0 \: \forall \: \phi$}
$$
\hat{P}_{\Psi}[ \phi] = \ket{\Psi}\inn{\Psi,\phi}
$$
where $\hat{P}_{\Psi} = \ket{\Psi}\bra{\Psi}$ is the \emph{projection operator}. Therefore $\Psi = [\alpha , \beta ]^T$ given $|\alpha|^2 +  |\beta|^2 = 1$ such that,

$$
P_{qubit} =
\left[
\begin{array}{cc}
P_{1,1} & P_{1,2}  \\
P_{2,1}  & P_{2,2}  \\
\end{array}
\right]
=
\left[
\begin{array}{cc}
|\alpha|^2 & \overline{\alpha} \beta \\
\alpha \overline{\beta}  & |\beta|^2   \\
\end{array}
\right]
$$
\subsubsection{Generalized Observables; Positive Operator Valued Measure}
With much of the language surrounding quantum mechanical standard formalism being considered as equivocal and found to be limited in it's applicability to explain experimental observations; this lead to the development of the \emph{Generalized Model of Quantum Theory}. In a generalized model, the \emph{generalized observable} of a Positive Operator Valued Measure (POVM) has an identity operator $\pi_b \geq 0$ and $\sum_{b} \pi_b = I$ (i.e. I is the unit operator) and a non-orthogonal decomposition $tr [\pi_b \pi_{b'} \neq 0]$ (Busch 2005 reference; the probability obtained from a random variable is obtained using the POVM or the quantum mechanical measurement probability $p_b$ of obtaining the $b$th event under the projection $\pi_{b}$, i.e. $\pi_b$ is a family of Hermitian operators, given by the unit-trace,
\begin{equation}
p_b = \text{tr}[ \rho \pi_b]
\end{equation}
where $\rho_b$  is a semi-positive \st{Hermitian} density operator from the density matrix $\mathbf{\rho} = \sum_b^k \rho_b p_b$; therefore the POVM within Quantum Mechanics replaces projector operators. The identity operators may also be represented as a combination of non-unique linear operators,
\begin{equation}
\pi_b = V_b^{*} V_b
\end{equation}
such that probability $p_b$ can therefore be represented by the unit-trace as,
\begin{equation}
p_b = \text{tr}[ \rho \pi_b] = \text{tr}[ V_b^{*} \rho V_b]
\end{equation}
Therefore the \emph{post-measurement} probability of being in state $b$,
\begin{equation}
\rho_b = \dfrac{V_b^{*} \rho V_b}{\text{tr}[ V_b^{*} \rho V_b]}
\end{equation}
\textbf{Connect the next section}
So in the Hilbert space, the fixed orthonormal basis $\ket{\alpha}$ consists of quantum states. Therefore the quantum system has a probability $p_b$ that it is in quantum state $\ket{\Psi_b}$ where in  $\inn{\Psi_b,\Psi_b} = 1$, such that at any given time proportional $p_b$  are in $\ket{\Psi_b}$; note that the quantum states are not constrained to being orthogonal. The set of states  spanned are defined by $p_b>0$ and $\sum_b^k p_b = 1$\footnote{In quantum mechanics, the self-adjoint operator $\hat{A}$ of an observable $A$ (i.e. $\hat{A}^T =\hat{A}$) is applied to a wavefunction to determine the expectation when in state $\ket{\Psi}$; wherefore if $\hat{A}\ket{\Psi}$ then a $\ket{\Psi}$ then $a$ is the  eigenvalue of $\hat{A}$ with eigenstate $\ket{\Psi}$}. The expectation therefore of the observable is given by,
\begin{equation}
\begin{split}
\inn{A} & = \sum_b^k p_b \inn{\Psi_b | A | \Psi_b}\\
& = \sum_b^k p_b \sum_{d,e}^l p_b \inn{\Psi_b | A} \inn{d| A | e} \inn{e | \Psi_b}\\
& = \sum_b^k \sum_{d,e}^l p_b \inn{e | \Psi_b} \inn{\Psi_b | A} \inn{d| A | e} \\
& = \sum_{d,e}^l \inn{e|\rho|d} \inn{d| A | e} \\
& = \text{tr} [\rho A ]
\end{split}
\end{equation}
where the density $\rho$ confers $\rho = \sum_b^k p_b\ket{\Psi_b}\bra{\Psi_b}$, $tr[\rho] = 1$, $\rho^T = \rho$, $\rho =  \sum_K \lambda_K\ket{K}\bra{K}$ such that $\inn{K|\rho|K}$, $0 \geq \lambda_K \leq 1$, $\sum_K \lambda_K = 1$; $\ket{\Psi}\bra{\Psi}$. (18.435/2.111 POVM Lecture//print25) Let us consider a two-dimensional space with basis consisting of two orthonormal vectors $[\ket{1},\ket{2}]$ such that the three-dimensional space may be represented by a basis consisting of two orthonormal vectors $[\ket{0},\ket{1},\ket{2}]$. We can consider state $\Psi$ such that in two-dimensions $\inn{1|\Psi|1} = 1/2$ and in three-dimensions $\inn{1|\Psi|1} = 1/3$. We therefore obtain the measurement of the quantum state in a reduced subspace $\pi_i$ with a projection length $\inn{\Psi |\pi_i| \Psi}$; this maps state $\ket{ \Psi }$ to $\pi_i$ at a length proportional to the square of the projection,
\begin{equation}
\dfrac{{\pi_i \ket{\Psi}}}{{\inn{\Psi |\pi_i| \Psi }^{1/2}}}
\end{equation}
We may therefore for a two-dimensional space (read print28quantum.pdf), give a Hamilton space $\mathpzc{H}$ $\ket{\Psi}$ that is a linear combination of vectors from that basis space. We  obtain the 'probability amplitudes'; $P_{1,2}[x]= |\Psi_{1,2}[x]|^2$, $P_{1}[x] = |\Psi_{1}(x)|^2$ and $P_{2}[x] = |\Psi_{2}(x)|^2$ where the quantum state $\Psi_{1,2}[x]$ therefore is $\Psi_{1,2}[x] =  \Psi_{1}[x] + \Psi_{2}[x]$. Therefore if we have a qubit in a two-dimensional space, its state is a unit vector in it's space with basis vectors $\ket{0}$ and $\ket{ 1}$. In the three-dimensional case where we add a basis vector $\ket{2}$ to the qubit, the orthonormal basis is in the three-dimensional space where each basis vector lies outside the subspace of the original qubit's:
\begin{equation}
\begin{split}
\dfrac{\sqrt{2}}{\sqrt{3}}\ket{0} + \dfrac{1}{\sqrt{3}}\ket{2} & \\
-\dfrac{1}{\sqrt{6}}|\ket{0} + \dfrac{1}{\sqrt{2}}\ket{1} + \dfrac{1}{\sqrt{3}}\ket{2} & \\
-\dfrac{1}{\sqrt{6}}|\ket{0} - \dfrac{1}{\sqrt{2}}\ket{1} + \dfrac{1}{\sqrt{3}}\ket{2} &
\end{split}
\end{equation}
therefore we may obtain the probability of the first vector,
$$
\bigr|(\dfrac{\sqrt{2}}{\sqrt{3}}\bra{0} + \dfrac{1}{\sqrt{3}}\bra{2})\ket{\Psi}\bigr|^2 = \bigr|\dfrac{\sqrt{2}}{\sqrt{3}}\inn{0|\Psi} \bigr|^2
$$
where we obtain it's quantum states as the unnormalized vector (check this - is this reall an unnormalized vector)  $\ket{e_1} = \sqrt{2}/\sqrt{3} \bra{0}$, $\bra{e_2} = - \sqrt{1}/\sqrt{6} \bra{0} + \sqrt{1}/\sqrt{2} \bra{1}$ and $\ket{e_3} = - \sqrt{1}/\sqrt{6} \ket{0} - \sqrt{1}/\sqrt{2} \bra{1}$. By definition for probabilities,
\begin{equation}
\begin{split}
& \sum_{i=1}^k |\inn{e_i|v}|^2 = 1 \\
& \sum_{i=1}^k \inn{v|e_i}\inn{e_i|v}  = 1 \\
& \sum_{i=1}^k \inn{v|e_i}\inn{e_i|v}  = 1 \\
\end{split}
\end{equation}
where $v$ is a unit vector. A necessary condition for the ensemble of $m_b$ matrices to be a POVM measurement is that there corresponding probabilities must always sum to 1. Let us consider $k$ unnormalized quantum states and $\sum_{i=1}^k \ket{e_i}\bra{e_i}$ is known to be Hermitian matrix $\mathbf{M}$ (an identity matrix) there are,
\begin{equation}
\begin{split}
& \bra{u}\bigr( \sum_{i=1}^k \ket{e_i}\bra{e_i} \bigr)\ket{u} = 1 \\
& |\bigr( \sum_{i=1}^k \ket{e_i}\bra{e_i}` \bigr)|  = I
\end{split}
\end{equation}
We can therefore state that for an identity matrix $M$, coordinate entry $(a,b)$ corresponds to row $a$ and column $b$, i.e. $\inn{a,e_b}$. Provided each  $i$-th row is orthonormal in a $k$-dimensional space, we obtain the inner product for row $j$ and row $k$ a $\sum_{b=1}^{c} \inn{j,e_b} e_b k$ such that for $k$ dimensions,
\begin{equation}
\begin{split}
& \inn{j |\bigr( \sum_{i=1}^k \ket{e_i}\bra{e_i} \bigr)| k} = 1 \\
& \inn{j,k}
\end{split}
\end{equation}
This therefore implies that for a projective measurement in a large space where it is restricted to a smaller space, it may be expressed as a  POVM of rank-1 by of unnormalized vectors $\ket{e_i}$ as,
\begin{equation}
\sum_{i=1}^k \bra{e_i}\ket{e_i}   = I
\end{equation}
therefore given a set of matrices $\gamma_i$ we have $\sum_{b=1}^{c} \gamma_b = I$ such that the probability is defined as,
$$
p_i = \inn{\Psi,\gamma_i,\Psi}
$$
therefore the we may show that the set of probabilities across $b$ matrices is  $1$,
\begin{equation}
\begin{split}
& \sum_{b=1}^{c} p_b = \sum_{b=1}^{c} \inn{\Psi,\gamma_b,\Psi} \\
& = \sum_{b=1}^{c} \text{tr} \ket{\Psi}\bra{\Psi} \gamma_b \\
& =  \text{tr} \ket{\Psi}\bra{\Psi} \sum_{b=1}^{c} \gamma_b \\
& =  \text{tr} \ket{\Psi}\bra{\Psi} I\\
& = 1
\end{split}
\end{equation}
which infers that each matrix $\gamma_b$ is a unique POVM measurement on a reduced subspace $\pi_b$. Therefore for a matrix $\gamma_b$ with the same rank as $\pi_b$ we must have,
\begin{equation}
\begin{split}
\gamma_b & = \sum_a \ket{e_{a,b}}\bra{e_{a,b}} =  \sum_a \lambda_a \ket{f_{ab}}\bra{f_{ab}} \leftrightarrow  \pi_b = \sum_a \ket{v_{a,b}}\bra{v_{a,b}}
\end{split}
\end{equation}
therefore there is a relationship between ${\Psi}$ and the quantum state in a higher dimension $\hat{\Psi}$,
\begin{equation}
\begin{split}
\inn{\Psi|\gamma_b|\Psi}=\inn{\hat{\Psi}|\pi_b|\hat{\Psi}}
\end{split}
\end{equation}

\newpage
\subsection{Von Neumann Entropy}
From a classical information theory standpoint, we must be able to obtain an expression for the reduction in the uncertainty of an observation $Y$ conditioned on $X$ as,
\begin{equation}
I[Y,X] = H[Y] - H[Y;X]
\end{equation}
such that given the conditional probability $p[y|x]$, the amount of information that may be obtained when transmitting across a noisy channel,
\begin{equation}
I[Y,X] = - \sum_{x \in K_X,y \in K_Y } p[y,x] \log \dfrac{ p[y|x]}{p[y]}
\end{equation}
However the Shannon entropy and mutual information measures are restricted to classical probability distributions. However we may adapt the Shannon entropy to the Quantum mechanical from of reference, by considering a density matrix as being analogous to a probability distribution. Therefore, instead of a classical probability relationship we may employ the quantum-information counterparts known as the quantum-mechanical \textbf{density amplitude probability} (check this), we arrive at a correct definition.  (extend this) For a given density matrix Von Neumann defines entropy as,
\begin{equation}
\mathpzc{S} [\rho] \equiv - \text{tr} [ \rho \: \log [\rho] ]
\end{equation}
where iff $\lambda_{\alpha}$ are the eigenvalues of the density matrix we may equivalently express the Von Neumann entropy as;
matrix Von Neumann defines entropy as,
\begin{equation}
\mathpzc{S} [\rho] \equiv - \sum_{\alpha} \lambda_{\alpha} \text{log} \lambda_{\alpha}
\end{equation}
therefore if we choose an orthonormal basis $\ket{\alpha}$ such that $\rho = \sum_{\alpha} \lambda_{\alpha} \ket{\alpha}\bra{\alpha}$,
$
\mathpzc{S} [\rho] = H[\mathpzc{Y}]
$
where $H[\mathpzc{Y}]$ is the Shannon entropy; note, that where the composition of the alphabet consists of orthogonal pure states the quantum source must be  equivalent to am information source such that,
$
\mathpzc{S} [\rho] = H[Y]
$.
\newpage

With a recent development of quantum computing, many machine learning work start to consider techniques of quantum mechanics such as quantum many-body physics[8, 9, 10],
phase-transitions [11], quantum control [12, 13], and quantum error correction [14, 15] because those methods can
be used for machine learning data analysis. A

The preparation of quantum states using short quantum circuits is one of the most promising near-term applications of small quantum computers, especially if the circuit is short enough and the fidelity of gates high enough.

The authors also implemented the phase estimation algorithm for quantum chemistry [32], which did not reach chemical accuracy due to the degradation of gate fidelity in the absence of error correction on their device.

However, a large scale of
quantum circuits cannot be faithfully employed upon the quantum computing platforms due to the lack of quantum
error correction [18, 19]. T

We have begun to see the adoption of quantum computing in Tensor models, where the tensor circuits are replaced by tensor network quantum circuity; in both supervised and unsupervised machine learning the quantum based tensor models have been shown to provide initially promising results [Exponential machines/Novikov, Supervised learning with tensor networks/Stoudenmore, Machine Learning by two-dimensional hierarchical tensor/Liu, Learning relevant features of data with multi-scale tensor netorks/Stoudenmire, Unsupervised generative modeling using matrix product states/Han]. Although Quantum Machine Learning as an application of quantum mechanics  has considerable potential, presently the limited number of qubits and the high incidence of error rates limits their fullscale adoption. One approach to deal with the limitation in the number of qubits is to adopt a novel architecture based on Tensor networks or scalable Tensor networks [ref Towards Quantum Machine Learning with Tensor Networks/Higgings].


\subsection{breal}
If instead of classical probability relationships we employ their quantum-information counterparts we may obtain the quantum-mechanical density amplitude probability, we arrive at a correct definition. We may be able to therefore extend our understanding of entropy to discipline of quantum information. We can consider a codeword at the source consisting of $z$ realizations chosen from an ensemble of quantum states belonging to an alphabet $\mathpzc{Z_Q}$ where each realization is according to a probability $p_Q[z]$.
	



%We can  like to generalize these considerations to quantum information. So let us imagine a source that prepares messages of n letters, but %where each letter is chosen from an ensemble of quantum states. The signal alphabet consists of a set of quantum states ρx, each occurring %with a specified a priori probability px. As we have al

\section{Chapter 2.1: Feature Importance Analysis}


In this section, we demonstrate that ML provides intuitive and effective tools for researchers who work on the development of theories. Our exposition runs counter to the popular myth that supervised ML models are black-boxes. According to that view, supervised ML algorithms find predictive patterns, however researchers have no understanding of those findings. In other words, the algorithm has learned something, not the researcher. This criticism is unwarranted.
Even if a supervised ML algorithm does not yield a closed-form algebraic solution (like, for example, a regression method would do), an analysis of its forecasts can tell us what variables are critically involved in a particular phenomenon, what variables are redundant, what variables are useless, and how the relevant variables interact with each other. This kind of analysis is known as “feature importance,” and harnessing its power will require us to use everything we have learned in the previous sections.


\section*{GPU}
You will get the best performance from GPU if your application is: – Computation intensive – Many independent computations – Many similar computations

\end{document}