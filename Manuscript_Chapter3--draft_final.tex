\documentclass[7pt]{article}
\usepackage[framemethod=TikZ]{mdframed}[2013/07/01]
\usepackage[T1]{fontenc}
\usepackage{afterpage}
\usepackage{framed}
\usepackage{physics}
\usepackage{amssymb}
\usepackage{soul}
\usepackage{changepage}
\newlength{\rulethickness}
\setlength{\rulethickness}{1.2pt}
\newlength{\rulelength}
\setlength{\rulelength}{15cm}
\usepackage[left=2cm, right=2cm, top=2cm]{geometry}
\usepackage[utf8]{inputenc}
\usepackage{setspace}
\usepackage{amsmath,amsthm,amscd}
\usepackage[colorlinks]{hyperref}
\usepackage{setspace}
\usepackage{xcolor}
\usepackage{xparse}
\usepackage{soul}
\usepackage{caption}
\usepackage{blindtext}
\usepackage{enumitem}
\usepackage{lipsum}% http://ctan.org/pkg/lipsum
\usepackage{setspace}
\usepackage{lscape}
\doublespacing


%%%% watch this video
%%%%https://www.youtube.com/watch?v=gtRLmL70TH0

%%%%also watch this
%%%%https://www.youtube.com/watch?v=p7Lv9GxigYU

%%%%%%%https://www.infoworld.com/article/2077257/floating-point-arithmetic.html

%%%%% http://web.mit.edu/hyperbook/Patrikalakis-Maekawa-Cho/node47.html

%%%https://northstar-www.dartmouth.edu/doc/solaris-forte/manuals/common/ncg/ncg_math.html

\begin{document}

A computer program, written in the 1970's and 1980's, would likely have been written for a mainframe which would have supported floating-point arithmetic; the introduction of the "floating-point unit" (FPU) or maths coprocessors to the microprocessor market initially provided the general purpose computer manufacturers from a technical design standpoint, a rudimentary bridge\footnote{The finite representation by a 16-bit integer or fixed-point, could be simply implemented using integer arithmetic and was therefore the convention on many early computers that did not support floating-point arithmetic. The primary limitations of fixed-point representation was a limited range and a tendency to generate errors for numbers near zero.}. To which the sound approach to development, a programmer would typically limit the use of coprocessors as they had latency which were three orders of magnitude larger compared to mainframes or supercomputers of the then conventional availability. The coprocessor posed as a stand-alone circuit or later built into the CPU: the maths coprocessor where available to an application would be responsible for calculating float-point arithmetic accurately, or if not absolutely right, limiting the likelihood the calculations were not therefore significantly wrong\footnote{In 1994, an entire line of CPUs by market leader Intel simply couldn't do their maths. The Pentium floating-point flaw ensured that no matter what software you used, your results stood a chance of being inaccurate past the eighth decimal point. The problem lay in a faulty maths coprocessor, also known as a floating-point unit. The result was a small possibility of tiny errors in hardcore calculations, but it was a costly PR debacle for Intel.}.  When the time came, the Institute of Electrical and Electronics Engineers (IEEE) Standard for Binary Floating-Point Arithmetic (Standard 754) (https://ieeexplore.ieee.org/document/8739150) was adopted by microprocessor manufacturers starting in 1980, were the gains to the users were not a cause simply because of presentation. The more innovative initial approaches to implementing floating-point calculations based on IEEE-754 initially adopted a shift-and-subtract algorithm but the more acute users noticed that compared to supercomputers the maths coprocessors provided pretty modest throughput. Pentium is a chipset initially \st{released} introduced in August of 1994 by the Intel corporation, instigated the use of a lookup table implemented as a programmable logic array: thus the speed and accuracy of floating-point calculations were to rival that modern of supercomputers. It is not only a standard for microprocessors but also specialized chips such as GPUs will also calculate floating-point arithmetic the CPUs way. This standard reach was eventually to extend up to supercomputers. A required part of all future IEEE-754 standards has been to define the minutiae around representation of how arithmetic results should be approximated in floating point- including approximation to what or by what to whom. Which level of precision and what degree of dynamic range were defined and adopted by array of the microprocessors. Ultimately all this would lead to increase sales of the microprocessors by widening the installed base by allowing portable code and this was to provide a degree of relief to system manufacturers: the standard due to it's level of precision is preferred for modelling all manner of phenomena including heart erythema, nuclear reactions, financial models, facial recognition and climatic patterns. But can you actually train users to analyse  extremely large data sets produced by these models, without teaching them how to develop tools in parallel computing that normally require development across decades of a career? But can you actually impart on application developers the intricacies of the Kahan SRT division tester (reference), in order to catch the Pentium-like division bugs of twenty-five years ago? Do you wish users to ponder over the flaw or even remember it? Understandably, what you do wish to impart to application developers from the IEEE 754 standard, the point of it all, is: to understand the form of floating-point arithmetic,  precision of representation, accuracy of a floating-point operation, stability of numerical algorithms and when parallel programming where the floating-point processing is applied. 

\section*{Floating-Point Data Representation}  

The setting up of the IEEE 754 Floating-Point Standard was an effort by the institute to allow  manufacturers to adopt a universal standard for the representation and operation of floating-point arithmetic calculations; this would thereby allow these features to be available in the market more affordably (reference "Why do we need floating point by Kahan"). The initial  draft standard had been adopted by most of the major computer and microprocessor manufacturers in 1984  including AT\&T, Intel, IBM, Apple and AMD; the final standard ratified in 1985 did not introduce significantly more requirements than this draft. This standard was adhered to by system manufacturers thereby committing to the standard, or at least 'conformity' to certain degree. This involved designing all their \st{your} future microprocessors towards \st{the} IEEE-754 or the most recent IEEE-754 iteration. The final revision of the standard settled upon as of the writing of this text had been revision IEEE 754-2019\footnote{\textit{IEEE-754-2019 - IEEE Standard for Floating-Point Arithmetic}}. Against these revisions, industry adoption of the standard has been a phenomenal success. As with many standards developed by Standards Committees, it is important for developers to understand the context and information that may be obtained from it.

According to the Microprocessor Standards Committees, in the current IEEE-754 standard a floating-point number must be representable numerically by bits sequences; this was defined by the earliest standard\footnote{\textit{IEEE-754-1985 - IEEE Standard for Binary Floating-Point Arithmetic}; the standard was specified for \textit{binary} floating-point arithmetic.} chaired by Kahan which was released in 1985. If devices require 'representations of floating-point data in the binary interchange formats', the values are uniquely identified in general by encoding using three-field types:
\begin{enumerate}
	\item sign (S) bit; 1-bit
	\item trailing significand (T) bits
	\item exponent (E) bits
\end{enumerate}
A single-precision format for a floating-point \st{of}, a 32-bit \underline{integer}, was mandated as part of the standard; whereas, a double-precision format for a floating-pointing \st{of}, a 64-bit \underline{integer}, with greater precision could by implemented by manufacturers at their discretion. This double-precision format allows for the increased dynamic range while maintaining adequate accuracy, at a cost however of twice the storage requirement and therefore bandwidth of single-precision per-component representation; it was  was designed to provide greater dynamic range and precision for applications mainly within scientific computing. The standard IEEE 754 when introduced \st{released} was considered prescriptive and doctrinal: an \textit{extended format} was assigned to each basic format, however the extended single-precision format would quickly become antiquated\footnote{The extended floating-point format was included to allow intermediate computations with wider dynamic range that would provide a final floating-point value in the basic floating-point format whilst allowing for 1) generally greater precision than that of the basic floating-point format and 2) a limited occurrence of underflows / overflows. In practice the extended precision double floating-point format is used in place of the single-precision.}
At this point, we would do \st{best} well to explain the interpretation of the three-field types;  the $S$ is used to interpret the sign of numerical values - $S=0$ indicating a positive number whilst $S=1$ indicates a negative, the $E$ representation determines the range of numbers that can be represented and $M$ which is \st{a little more complex to explain but} is the part of the floating-point number consisting of its significant digits although it's size may vary such that a higher precision representation can opt for a trailing significand in a reduced form.  The IEEE single-binary format specifies 1- sign bit, 8- exponent bits and 23- trailing significand bits; the greater dynamic range offered by the IEEE double-binary format species 1- sign bit, 11- exponent bits and 52- trailing significand bits.
\subsection*{Sign (S) -bit representation}
The consequences of the value in a sign bit may be readily generalized. Positive-valued signs simplified as $S= 0$ can be explained in this context: when any numerical value is raised to the 0th power, including negative numbers, will total 1; hence $S=0$, positive is assigned. Excluding positive-valued instances, we observe that when -1 is raised to the 1st power, the numerical value remains negative; multiplying any number by -1 will always change the sign of others. Broadly generalizing therefore, as there are many exceptions, with -1 as a prefix raised to the power of its sign we may produce a floating-point. All the numerical representations of the Floating-Point Standard in the interchange format, with notable exception conditions and with appropriate default handling, when received in whatever way such that a floating point number represented by a double $(T_{*}, b, E)$ would be encoded down:
\begin{equation}
x = T_{*} \times (b^{E-bias})
\end{equation}
where $b$ the \st{constant base} radix \st{is set at 2 in binary} and $E$ is always a signed integer. The standards adoption of an \textit{sign-and-magnitude} representation for the significand  allows a floating-point number to be presented by a triplet $(S, T, b, E)$\footnote{$S \in \{0,1\}$ }, such that:
\begin{equation}
x = (-1)^{S}\times 1 . T \times (b^{E-bias})
\label{eq:IEEE754}
\end{equation}

\subsection*{Trailing significand (T) -bits representation}
%%%https://www.sciencedirect.com/topics/computer-science/mantissa
The use of the format in (\ref{eq:IEEE754}) creates a paradox. Many of the most reasonable representations of a numerical values, to it would involve derivations where treatment of the trailing significand bits may naturally differ from a $1.T$ format, limiting the use of many floating-point representations that would otherwise benefit due to the \st{standards} requirements open limitations. In IEEE-754, for example, it is acceptable to present $0.25_D = 1.0_B \times 2^{-2}$ using a trailing significand bit pattern where $T=1.0_B$ but $0.25_D = 10.0_B \times 2^{-3}$ where $T=10.0_B$ and $0.25_D = 0.1_B \times 2^{-1}$ where $T= 0.1_B$, but neither is according to the form $1.T$ despite there being no loss of precision. It is right to ask whether $"1."$ might be passed over from the representation, due to the inference that $".XX"$ is $1.XX$ would be known and there	 being a scarcity in trailing significand bits, in a floating-point computation. Thus, what is the best answer for those wishing to adhere to the IEEE-754 yet conserving to conserve their bits? It is generally accepted in IEEE format where the trailing significand is binary, to use the 2-bit trialling significand representation to mean its 3-bit standard form; the left-most bit defaults to $"1."$ in a scheme like this. This restriction in formatting is termed a \textit{normalized} floating-point\footnote{For floating-point arithemetic the number formatting is considered normalized where the dynamic range is given as: $$\dfrac{1}{radix} \leq significand < 1$$}, as for binary the leading digit of the signficand will always required to always be 1; normalized floating-point has hidden or implies that the most significant trialling signficand bit is 1; therefore a signficand is effectively $(T+1)$ bits throughout most of the representations.

\subsection*{Exponent (E) -bits representation}
The exponent bits, the number of bits assigned to $E$, is used to determine the range of a numerical outlook. The interaction between the exponent $E$ and its base of 2 in  floating-point numbers, determines a magnitude; for example, a large positive $E = 32$ would correspond to the large floating-point number $>2^{32}$ and a large negative $E = -32$ would correspond to a significantly smaller number $2^{32}$ - the exponent allowing for a substantially differing points of representation. The magnitude or accuracy \st{precision} may be more important depending on the characteristics of the data; for example, biological sciences might require large floating point values to capture growth, or a high degree of accuracy \st{precision} when examining characteristics which are weak or fragmented. It also enables through IEEE-754 a more extensive range of numbers than representations as standard integers, decimals and so forth.

The standard, IEEE-754, in its specification implements \textit{excess encoding} or bias representation for $E$;  given $E$ bits to encode the exponent $E$, an excess encoding is obtained by adding $(2^{E-1} - 1)_B$ to a two's complement representation for the exponent. The standards triplet thus becomes as triplet $(S, T, b, E)$:
\begin{equation}
x = (-1)^{S}\times 1 . T \times (b^{E-1})
\end{equation}
The excess-three code is a binary-coded-decimal (BCD) and numerical representation which is inherently biased. We consider an example of a 8-bit integer, consisting of 1-sign bit, a 3-bit exponent and a 4-bit trailing signficand. A decimal number $10.23$ would have a BCD code $1010$ and it's excess-three code would be obtained by adding $+3$ to each digit to get the equivalent excess-three code - in this case $0100 0011.0101 0110$. The number -3 being close to -2 in our 3-bit encoding scheme, within our hypothetical bit integer format of 8. The decimal $-0.25_D$ may therefore be represented in our 8-bit format as $-0.25_D = (-)1.0_B \times 2^{-2} = 1 001 0000 $, where $S=1$, $E = 001$ and $T = (1).0000$
\footnote{
Consider that $-0.25_D$, \\ a.  A conversion of the numeric value to binary scientific notation: $ -1.0_B \times 2^{-2}$, \\
b. We may obtain the triplet bits of our 8-bit format: \\
     	1. We obtain $S = 1$ as the number is negative, \\
      2. With excess-3 encoding we add 3 to the exponent and add store the code generated for the three exponent bits, \\
      i.e. $S = (-2 + 3)_{D} = 1_{D} = 001_{B} \equiv E = 001$. \\
      3. The significand bits are  obtained as the first four bits following an implied 1, \\
       i.e. $-1.0_B \times 2^{-2} = -1.0000_B \times 2^{-2}$, i.e. $T = 0000$. \\
We therefore obtain $1 001 0000$ in excess-3 encoding. 
}

\st{This excess encoding representation makes it possible to represent floating-points as either positive or negative.}


\afterpage{
	\begin{figure}
		\label{fig:excess7}
		\centering
		\begin{tabular}{|c|c|c|}
			\hline
			\textbf{Decimal value} & \textbf{Twos-complement representation} & \textbf{Excess-7 coded representation}  \\
			\hline
			\hline
			-7 &  1001 & 0000\\
			\hline
			-6 &  1010 &  0001\\
			\hline
			-5 & 1011 &  0010\\
			\hline
			-4 & 1100 &  0011\\
			\hline
			-3 & 1101 & 0100\\
			\hline
			-2 & 1110 &  0101\\
			\hline
			-1 & 1111 &  0110\\
			\hline
			0 & 0000  & 0111\\
			\hline
			1 &  0001&  1000\\
			\hline
			2 & 0010 &  1001\\
			\hline
			3 & 0011 &  1010\\
			\hline
			4 & 0100 &  1011\\
			\hline
			5 &0101  &  1100\\
			\hline
			6 & 0110 &  1101\\
			\hline
			7 & 0111 & 1110 \\
			\hline
			IEEE special bit pattern & 1000  & 1111 \\
			\hline
			\hline
		\end{tabular}
		\caption{Excess-seven encoding of decimal numbers indicates that the values returned are sorted. }
	\end{figure}
	\clearpage
}

This excess encoding representation was intended to allow direct comparison of signed numbers \st{by an unsigned comparator} - the use a signed comparator is more costly; other representations such as two's-complement representation and sign-and-magnitude representation were possible, but the comparison of a floating-point would best suits excess representation for exponents (hardware book 224,126). There is only one unsigned comparator required to compare signed floating-point numbers when excess representation is used on the exponents encoding. The penalty for negatively signed binary numbers under twos complement, would require a signed comparator, as  using an unsigned comparator will not be passed-on or caught. This behaviour leaks across the comparator and would require additional support. The binary numbers in the excess-seven coded bits however of table \ref{fig:exces7}, indicates that $-7_D$ has the least binary integer value of $0000$.  Comparing floating-point exponents along a sequence of excess-seven coded bits will correctly yield an accurate order; for instance, a comparison of  $-7_D$ and $0_D$ in the excess-seven code accordingly assigns order in terms of the largest number to  smallest. This is desirable from a manufacturers standpoint, as it is a relatively straightforward feature to implement. Excess-coded floating-point representation has an additional gratifying feature useful during implementation: a successor of a floating-point number is obtained by considering the binary representation as an integer and then incrementing the integer; this holds true for all positive floating-point numbers.

For completeness, IEEE floating-point binary format does not assign bit-integers of an exponent is a series of all $1's$,  i.e. as in $1111$ of Table  \ref{fig:excess-seven} , but they are is handled as a special-case. Though considered  an independent special bit pattern in the IEEE-754 binary format, the representation of the special bit pattern is dependent on the value of the trailing significand $T$; with a value of $T=1$ the \st{exception} special-case is handled as Not-a-Number (NAN) and  with a value of $T=0$ this \st{exception} special case is handled as infinity - the dynamic range of representation itself is however dependent on the level of precision adopted in numbering. Of the 64 bits in IEEE double-binary format an additional 52 bits are assigned to the trailing signficand and 11 bits to the exponent compared to 23 bits for the significand and 8 bits for the exponent in the single-binary format; the precision therefore achievable through normalized numbers in IEEE double-binary format is $2^{52}$ compared to the single-binary format's $2^{23}$.
%%%https://northstar-www.dartmouth.edu/doc/solaris-forte/manuals/common/ncg/ncg_math.html

\begin{landscape}
\afterpage{
\begin{table}
	\centering
\begin{tabular}{|c|l|l|l|l|l|}
\hline 
\textbf{Excess encoding, E} & \textbf{Signif., T} & \textbf{Sign, S} & \textbf{Single-Format(hex)} & \textbf{Double-Format(hex)} & \textbf{Handling}  \\
\hline
\hline
1111...111 & non-zero  & & 7fc00000 & 7ff80000 00000000 & nan (not-a-number) \\
\hline
1111...111 & zero & 1 &7f800000 & 7ff00000 00000000 & +INF (positive-infinity)\\
\hline
1111...111 & zero & 0 & ff800000 & fff00000 00000000 & -INF (negative-infinity) \\
\hline
0000...000 & zero & 0  & 80000000 &80000000 00000000 & signed zero (-0.0) \\
 &  & 0  & 00000000 &00000000 00000000 & (+0.0)  \\
\hline
0000...000 & non-zero & 0  & 007fffff & 000fffff ffffffff & denormalized / sub-normal \\
0000...000 &  & 0  & 00000001 & 00000000 00000001 &  \\
\hline
\end{tabular}
\caption{Special bit patterns in IEEE binary floating point format. The details behind special bit patterns are revealing. These situations were described as ones in which the floating-point number did not fit into the range defined within the standard or there as another type of error in the execution of an application. Infinity ($\infty$) representations, negative infinity or positive infinity, are generated when a very large number is divided by a very small number, leading to a special-case. The not-a-number (NaN) representations, are generated by invalid operations such as $0/0$, $\infty - \infty$ or $\sqrt{-1}$ leading to a special-case. The IEEE-754 format specifies two types of NaNs: signalling-NaNs which are represented by the most significant bit of the significand set (s  11111111  1xxxxx...xxxxx)  and the quiet-NaNs (s 11111111 0xxxxx...xxxxx) with the most significant bit of the significand not set. The quiet-NaNs occur when an invalid operation occurs during floating-point arithmetic but will allow the output to propogate through their standard operations; they do not signal exceptions and thereby allowing for diagnostic examination. The signaling-NaNs when an invalid operation occurs that must cause execution of the program to halt, thereby generating an exception. The exception classes were defined in the IEEE-754 standard; exceptions were to be only called with the results of an operation; the status-flags of exceptions were to include invalid operations, dividebyzero operations, underflow operations, overflow operations and inexact operations.  One of most common problem areas where [signalling] NaN are used is with detecting uninitialized variables; many programming languages allow uninitialized data to be assigned NaN and thereby detected by during  code compiling. }
\end{table}
	\clearpage
}
\end{landscape}

 (Cherkassky et al. (1994)):\\
\begin{enumerate}
\item Model: Pattern recognition favours interoperable models best suited  to structured modelling frameworks; machine learning has its chief objective as prediction. 
\item Complexity: traditionally machine learning has required large amounts of training data and high complexity to generalise its predictions; pattern recognition has tended to involve small data sets and lower complexity models. 
\item Data processing: models in pattern recognition have tended to require training of whole data sets, referred to as a \textit{batch} methods; machine learning adopts iterative processing termed \textit{on-line} methods which may be trained on portions of a larger datasets or using cross-validation techniques. 
\item Training speed: machine learning methods favour an iterative processing which tends to utilise much more tedious training regimes requiring multiple presentations of the data such as gradient descent, pattern recognition comparatively is much faster. 
\item Use complexity: As pattern recognition techniques extract information using a batch method they will tend to be more honourous to apply. Machine learning methods tend to be computationally for users to apply and simpler to interpret as they use smaller chunks of data.
\item Robustness: Machine learning estimators are typically more robust than the pattern recognition counterparts\footnote{Confidence intervals are routinely provided in statistical methods but are usually lacking in most machine learning studies.}.  
\end{enumerate}

\section*{Arithmetic Accuracy}

One of the most consistent problems relating to floating-point arithmetic when IEEE floating-point format was introduced was accuracy. This had often emerged, along with memory and cost constraints, during the introduction of personal computers. During the design of the IEEE-754, precision had been accounted for by the width of the trailing significand bits by the working group. The accuracy level of a floating-point operation however is considered captured by the number of operations that is to be carried out on a floating-point number; the error level however was considered low as the floating-point operands \st{numbers} were to be represented by 32 bits. This accuracy is measured in terms of the \textit{maximal error}, the higher the accuracy required the smaller the observed errors, for each operation. To this end, in floating-point arithmetic  'errors' are introduced whenever a number fails to be definitely represented on floating-point bit patterns. The error of each operation would be caused by any approximation in IEEE-754. By representing floating-point numbers, in the original standard, using a 32 bit significand an error would occur when the result representation required any additional bits. The nature of the IEEE-754 binary floating point standard, requires that before an operation the exponents must be equal; therefore, where two operands are of different magnitude as in \ref{right-hand-calc}, the exponents of the smallest operand is adjusted upward whilst the bits of the signficands are accordingly shifted right, requiring additional bits than the standards representation of floating-points. The most significant error would be propagated in this operation due to bit shifting in the smallest operands significand.

Yet even this error limit proves to be too high, to significantly reduce the level of error which propagated, particularly in scientific computing.


\begin{figure}
\label{right-hand-calc}	
\end{figure} 

 


%% https://scikit-learn.org/stable/tutorial/machine_learning_map/index.html

 \end{document} 